\documentclass{article}
\usepackage[a4paper,left=3cm,right=3cm,top=1cm,bottom=2cm]{geometry}
\usepackage{amsmath}
\usepackage{amssymb}
\usepackage{hyperref}
\usepackage[russian]{babel}

\usepackage{tikz-cd}
\usepackage{array}
\usepackage{graphicx}
\newcommand\mapsfrom{\mathrel{\reflectbox{\ensuremath{\mapsto}}}}
\setlength{\parindent}{0mm}

\usepackage{fontspec}
\setmainfont{Linux Libertine O}
\usepackage{unicode-math}
\setmathfont{Cambria Math}

\begin{document}
\title{
\textit{\small{Геогий Потошин, 2025}}\\
\vspace{0.3ex}
\textit{\huge{Французский для математиков}}\vspace{1ex}
}
\date{\vspace{-5ex}}
\maketitle

\section{Введение}
Эта брошюра для тех, кто хочет читать французские книги по математике. Здесь
будут рассказаны стандартные обороты. Тренировка произношения тоже важна. В
общих словах нужно кортавить, ставить ударение на послединий слог. Произношение
звуков нужно выносить как можно ближе к губам. Также во французском в отличие от
русского необходимо правильно произносить все гласные, так как это меняет смысл
и насители вас просто не поймут, так как в их языке нет такого феномена и их
мозг не натринерован определять слова с вариацией произношения. Для тренировки
произношения его можно слушать на таком-то сайте где если вырезки произношения
французов из видео, а также записывать себя прослушивать то что сказали и
корректировать. Прои произношении слов на французском звуки должны перетикать
из одного в другой. В большинстве случаев в граница между звуками смазывается.

\section{Английский и Французский}
Как известно Франция в средневековье завоевала Англию и процентов 40 слов
Английского имеют французское происхождение. Поэтому есть феномен перекрестного
понимания. Написание слов конечно же отличается, но в основном это выражается
в окончаниях. Во французском на конце часто добавляется \emph{-e} по сравнению с
английским и английское окочание \emph{-er} заменятеся на \emph{-re}.

\section{Глаголы}
Основным глаголом как и в английском является глагол быть на французком
\emph{être}. Он читается \emph{этр} c кортавой \emph{р}. Звук \emph{т} находится
в слабой позиции и он наименее слишим из этих трёх. Во французком буква простая
\emph{e} в конце слова никогда не произносится. Её конечно же читал в прошлом и
её отзвук слышин в некоторых акцентах, например в Парижском. В таком случае
этот глагол читается следующим образом. Звук \emph{э} произносится все время
с затуханием под конец, интонация (высота звука) тоже убывает, а согласные
произносятся поверх. Ударение ставится на первое \emph{э} так как остальное –
это её отзвук. Он в основном встерчается в трех формах. Если глагол используется
в единственом  числе третьего лица, то есть когда мы например говорим об
обьектах или структурах, то мы используем форму \emph{est}. Она всегда
произносится как \emph{э}. Эта форма очень похожа по написанию на немецкую форму
\emph{ist} глагола быть \emph{sein} и чуть меньше, но похожа на английский
глагол \emph{is}. Например фраза 'группа проста' на французком будет \emph{groupe
est simple}. Вторая форма – это форма множественного числа третьего лица
\emph{sont}, она читается как \emph{сон} причем \emph{n} произносится как нчание
(н + мычание), в таком случае говорят, что она нозальная. Её мы используем для
многих объектах и в русском она соответствует глаголу третьего лица,
множественного числа \emph{суть}. Третья форма имеет значение пусть и ставится
перед определяемым объектом. Эта форма пишется \emph{soit} и читается \emph{суа}
при произношении одна гласная должна перетекать в другую. У этой формы есть
множественное число! Оно пишется \emph{soient} и читается \emph{суайон}, где
\emph{н} вновь нозальная. Как вы можете заметить, окончания \emph{-nt} является
окончанием глаголов множественного числа третьего лица. Примером фраз будет
'пусть $G$ просто' \emph{soit $G$ simple} или 'пусть $G_i$ просты' \emph{soient
$G_i$ simples}. Как вы можете заметить окончание $-s$ означает множественное
число как у существительных, так и прилагательных, но оно не читается.

\section{Векторные пространства}
\section{Анализ}
\section{Группы}
\section{Кольца}


\end{document}
