In this section, we re-examine $\mathcal C^1$ $n$-dimentional submanifold $M$
of $\mathbb R^{n+m}$ to observe several properties of standart constructions.
This will then motivate the definition for analogus constructions for measures.

\vspace{2ex}
We can define a gradient $\nabla$ through its representation
\[\langle \nabla f, \cdot\rangle=Df\]
Then we define a gradient $\nabla^M$ associated with $M$ by
\[\nabla^M f(x) := \pi_{T_xM}(\nabla f(x))\]
and if we have an orthogonal basis $(\mathbf w_i)$ of $T_xM$
\[\nabla^M f(x) = (\mathbf w_1,\ldots,\mathbf w_n)\left(\begin{array}{c}D_xf(\mathbf w_1)\\ \vdots\\D_xf(\mathbf w_1)\end{array}\right)=(\mathbf w_i)(D_xf(\mathbf w^i))\]
for an orthogonal basis $(\mathbf w_i)$ of $TM$. If we take that for a definitions, then
using matrix notation, it's also easy to show that this notion is independed of the chosen basis,
which is already evident from coordinateless defintion. If we have another
orthogonal basis $(\mathbf w_i)=(\mathbf u_i)P$, where $P$ is orthogonal, 
then $(\mathbf w^i)=P^t(\mathbf u^i)$ and thus $(\mathbf w_i)(Df(\mathbf w^i))
=(\mathbf u_i)P(DfP^t(\mathbf u^i))=(\mathbf u_i)PP^tDf(\mathbf u^i)=(u_i)Df(u^i)$

\vspace{2ex}
If for two vectors $a,b$ by $ab$ we right their scalar product, then we can
introduce diveregence of $X:M\rightarrow\mathbb R^{n+m}$ on $M$ by chosing
a basis $(\mathbf w_1,\ldots,\mathbf w_n)$ of $T_xM$ and setting
\[\text{div}_MX(x)=(\mathbf w_i)(\nabla^MX^i(x))\]
where $X=(X^i)_{i\in\llbracket1,n+m\rrbracket}$ are coordinats in orthogonal
extension of $(w_i)$. We will need a following theorem

\vspace{2ex}
\textbf{Theorem:} \textit{Let $M\subset \mathbb R^{n+m}$ be a $n$-dimentional
$\mathcal C^1$ bounded submanifold-with-boundary. Then we have a following result}
\[\int_M \text{div}_MXd\mathcal H^n=\int_{\partial M}X\nu d\mathcal H^{n-1}\;\textit{for every}\;X\in\mathcal C^1(M,TM)\]
\textit{where $\nu$ is an orthogonal unitary vector to the boundry pointing outwards.}

\vspace{2ex}
We will now present a proof for the $\mathcal C^2$ case, as this regularity allows
us to utilize coordinates and integrate by parts. For this section, we assume that $X(x)\in
T_xM$. Let us the consider an orthonormal basis $(\mathbf w_i)$ of $T_xM$
which is extended by vectors $(\mathbf w_i')$ to an orthonormal basis of
$\mathbb R^{m+n}$. Futhermore, consider curvelinear coordinates $(c^i):M\cap
U\rightarrow \mathbb{R}^n\cap V$ ($\mathcal C^2$-diffeomorphism), to which we can
associate a standart basis 
\[\mathbf{c}_i=\mathbf{w}_j\frac{\partial w^j}{\partial c^i}+\mathbf{w'}_j\frac{\partial w'^j}{\partial c^i}\]
\begin{center}
\includegraphics[scale=0.2]{maps.png}
\end{center}
which consists of tangents to the coordinate curves. Thus, $(\mathbf c_i)$ form a
basis, whose vectors vary with position and are associated with coordianets. In coordinate-free
language this basis is an image of standart basis $\mathbf e_i$ of $\mathbb R^n$
by $(D_x(c^i))^{-1}$. By $(w^i)$
here, we denote the standard coordinates associated with basis $(\mathbf w_i)$. In the
following discussion all expressions are evaluated at a point x; moving to a
different point necessitates selecting a new tangent basis. Given that $(\mathbf c_i)$
varies smoothly ($\mathbf C^1$), we can differentiate them and
naturally define the following coefficients, called Christoffel symbol:
\[\mathbf c_k\Gamma_{ji}^k=\pi_{T_x M}(\frac{\partial\mathbf{c}_i}{\partial c^j})\]
Indeed, we observe symmetry in lower indices since
\[\pi_{T_xM}(\frac{\partial\mathbf{c}_i}{\partial c^j}) = \frac{\partial}{\partial c^j}(\mathbf{w}_l\frac{\partial w^l}{\partial c^i})
=\mathbf{w}_l\frac{\partial^2 w^l}{\partial c^j\partial c^i}=\mathbf{w}_l\frac{\partial^2 w^l}{\partial c^i\partial c^j}
=\Gamma_{ij}^k\]
Also in a new basis $(\mathbf{c}_k)$ we have scalar product, which is usually denoted by
\[\langle\mathbf{c}_l,\mathbf{c}_k\rangle=g_{lk}=\frac{\partial w^i}{\partial c^l}\frac{\partial w^i}{\partial c^k}\]
which is also symmetric. If we then rewrite the equality for $\Gamma$, we get
\[\mathbf{w}_j\frac{\partial^2 w^j}{\partial c^j\partial c^i}=\mathbf{c}_k\Gamma_{ji}^k
=\mathbf{w}_j\frac{\partial w^j}{\partial c^k}\Gamma_{ji}^k\]
The metric defines an isomorphism $\phi$ from our space $T_xM$ to its dual space $T_xM^*$ by
\[\phi(\mathbf{c}_i):=\langle\mathbf{c}_i,\cdot\rangle=g_{lm}\mathbf{c}^{*l}\otimes\mathbf{c}^{*m}\mathbf{c}_i=g_{li}\mathbf{c}^{*l}\]
The components $g_{ij}$ form a matrix representation of $\phi$ in local coordinates.
Consequently, the matrix of $\phi^{-1}$ is $(g_{ij})^{-1}$. Associated with this
ismorphism, we can then construct a dual product by
\begin{align*}
    \langle\cdot,\cdot\rangle^*:&=\phi^{-1}(\langle\cdot,\cdot\rangle)=\phi^{-1}(g_{lm}\mathbf{c}^{*l}\otimes\mathbf{c}^{*m})
=g_{lm}\phi^{-1}(\mathbf{c}^{*l})\otimes\phi^{-1}(\mathbf{c}^{*m})\\
    &=g_{lm}(g^{-1}_{lr}\mathbf{c}_r)\otimes(g^{-1}_{mk}\mathbf{c}_k)
    =\delta_m^r\mathbf{c}_r\otimes g^{-1}_{mk}\mathbf{c}_k=g^{-1}_{rk}\mathbf{c}_r\otimes \mathbf{c}_k
\end{align*}
Usually we  write $\langle\cdot,\cdot\rangle^*=g^{mk}\mathbf{w}_m\otimes \mathbf{w}_k$.
Let $X'$ denote $X$ rewritten in new coordinates, i.e.
\[\mathbf{X}(w^1,...,w^n)=\mathbf{X}'(...c^i(w^1,...,w^n)...)\]
Now we take a partial derivative of both sides. In this step, it is
crucial that $X$ has values tangent space, as this allows to write $X$ in the
curvilinear basis
\begin{align*}
\frac{\partial}{\partial w^j}\mathbf{w}_iX^i&
=\pi_{T_x M}(\frac{\partial}{\partial w^j}\mathbf{c}_pX'^p)
=\pi_{T_x M}(\frac{\partial}{\partial c^k}\mathbf{c}_pX'^p)\frac{\partial c^k}{\partial w^j})
    =(\mathbf{c}_p\frac{\partial X'^p}{\partial c^k}+\pi_{T_x M}(\frac{\partial\mathbf{c}_p}{\partial c^k})X'^p)\frac{\partial c^k}{\partial w^j}\\
&=(\mathbf{c}_p\frac{\partial X'^p}{\partial c^k}+\mathbf{c}_l\Gamma^l_{pk}X'^p)\frac{\partial c^k}{\partial w^j}
=(\mathbf{c}_p\frac{\partial X'^p}{\partial c^k}+\mathbf{c}_p\Gamma^p_{lk}X'^l)\frac{\partial c^k}{\partial w^j}
=\mathbf{c}_p(\frac{\partial X'^p}{\partial c^k}+\Gamma^p_{lk}X'^l)\frac{\partial c^k}{\partial w^j}
\end{align*}
Then we can calculate divergence formula at $x$ by
\begin{align*}
\text{div}\mathbf{X}&=\mathbf{w_j}\cdot\frac{\partial}{\partial w^j}\mathbf{w}_iX^i
=\mathbf{w}_j\cdot\mathbf{c}_p(\frac{\partial X'^p}{\partial c^k}+\Gamma^p_{l,k}X'^l)\frac{\partial c^k}{\partial w^j}
=\mathbf{w}_j\cdot\mathbf{w}_i\frac{\partial w^i}{\partial c^p}(\frac{\partial X'^p}{\partial c^k}+\Gamma^p_{l,k}X'^l)\frac{\partial c^k}{\partial w^j}\\
&=\frac{\partial c^k}{\partial w^j}\frac{\partial w^j}{\partial c^p}(\frac{\partial X'^p}{\partial c^k}+\Gamma^p_{l,k}X'^l)
=\frac{\partial c^k}{\partial c^p}(\frac{\partial X'^p}{\partial c^k}+\Gamma^p_{l,k}X'^l)
=\frac{\partial X'^p}{\partial c^p}+\Gamma^p_{l,p}X'^l
\end{align*}
For the next part we are differentiating $g_{lm}$
\[\frac{\partial}{\partial c^k}g_{lm}=(\frac{\partial}{\partial c^k}\frac{\partial w^i}{\partial c^l})\frac{\partial w^i}{\partial c^m}
+\frac{\partial w^i}{\partial c^l}(\frac{\partial}{\partial c^k}\frac{\partial w^i}{\partial c^m})
=\frac{\partial w^i}{\partial c^r}\Gamma^r_{kl}\frac{\partial w^i}{\partial c^m}+
\frac{\partial w^i}{\partial c^l}\frac{\partial w^i}{\partial c^r}\Gamma^r_{km}
=g_{mr}\Gamma^r_{kl}+g_{lr}\Gamma^r_{km}\]
Now lets consider a following quantity
\[\frac{\partial g_{kl}}{\partial c^m}+\frac{\partial g_{km}}{\partial c^l}-\frac{\partial g_{ml}}{\partial c^k}
=g_{kr}\Gamma^r_{ml}+g_{lr}\Gamma^r_{mk}+g_{kr}\Gamma^r_{lm}+g_{mr}\Gamma^r_{lk}-g_{mr}\Gamma^r_{kl}-g_{lr}\Gamma^r_{km}=2g_{kr}\Gamma^r_{ml}
\]
And if we multiply both sides by $1/2g^{ak}$ we get
\[g^{ak}g_{kr}\Gamma^r_{ml}=\delta_r^a\Gamma^r_{ml}=\Gamma^a_{ml}=\frac{1}{2}g^{ak}(\frac{\partial g_{kl}}{\partial c^m}+\frac{\partial g_{km}}{\partial c^l}-\frac{\partial g_{ml}}{\partial c^k})\]
Now lets take Cristoffel symbol from divergence formula and develop it
\[\Gamma^p_{pl}=\frac{1}{2}g^{pk}(\frac{\partial g_{kl}}{\partial c^p}+\frac{\partial g_{kp}}{\partial c^l}-\frac{\partial g_{pl}}{\partial c^k})
=\frac{1}{2}g^{pk}\frac{\partial g_{kp}}{\partial c^l}\]
Now knowing that $\partial_k(\det A)=\det(A)\text{tr}(A^{-1}\partial_k A)$ and
writing $g=\det(g_{ij})$ we have
\[g^{pk}\partial_l g_{kp}=\text{tr}((g_{kp})^{-1}\partial_l(g_{kp}))=\frac{\partial_l g}{g}\]
And since $\partial_l\sqrt g=\frac{1}{2\sqrt g}\partial_l g$ we have
\[\Gamma^p_{pl}=\frac{1}{\sqrt g}\partial_l\sqrt g\]
and finally we can rewrite divergence formula as
\[\text{div}\mathbf X=\partial_p X'^p+\frac{1}{\sqrt g}\partial_l\sqrt g X'^l
=\frac{1}{\sqrt g}\sqrt g\partial_p X'^p+\frac{1}{\sqrt g}\partial_p\sqrt g X'^p
=\frac{1}{\sqrt g}\partial_p(\sqrt g X'^p)
\]

\vspace{1ex}
\textbf{Proof:}
Since $M$ is compact we can find a finite open cover $U_i$ of
$M$ such that at every $U_i$ we find a local coordinate system. We can associated
a $\mathcal C^\infty$ partition of unit $(\psi_i)$ on $M$ associated to $U_i$ such
that $\psi_i\in\mathcal C^\infty_c(U_i)$ and $\sum\psi_i=1_M$. We shall write
$O_i=U_i\cap M$. And by liniarity of equation we want to prove, we can prove it
jsut for $\psi_iX$, or without loss of generality we will write just $X$. Let
coordinates take values in $V_i$

\vspace{1ex}
If patch $U_i$ is disjoint from $\partial M$, then $V_i$ are open and $X$ is of
a compact support inside $V_i$ and we can integrate by parts
\[0=\int_{V_i}\partial_j 1X^j\sqrt gdc=-\int_{V_i}1\partial_j(X^j\sqrt c)=
-\int_{V_i}\frac{1}{\sqrt g}\partial_j(X^j\sqrt g)\sqrt gdc=-\int_{U_i}\text{div}_M Xd\mathcal H^n\]
Thus inner patches have no contribution. Lets take a patch $U_i$ that intersects
boundary. By definition of submanifold-with-boundary we can introduce coordinates
$c^i$ such that $\{c_n=0\}=\partial M\cap U_i$ and $c_n$ has values only in
$(-\infty,0]$. This time $X$ does not have a compact support inside $V_i$ and
thus in integration by parts we have the 2 terms
\[0=\int_{V_i}\partial_j 1X^j\sqrt gdc=-\int_{U_i}\text{div}_M Xd\mathcal H^n+\int_{\mathbb R^{n-1}}X^n(c',0)\sqrt{g(c',0)}dc'\]
And as we can chose such coordinates, that $X^n(c',0)=\nu\cdot X(c',0)$ we have
\[\int_{U_i}\text{div}_M Xd\mathcal H^n=\int_{\partial M}\nu\cdot Xd\mathcal H^{n-1}\]

\vspace{2ex}
\textbf{Definition:} \textit{Let $\mathbf{v}_i$ be an orthonormal basis of $T_yM$ which
is $C^1$ funciton of $y$. Then we can define \textbf{the second fundamental form}
$B_y:T_yM\times T_yM\rightarrow (T_yM)^\perp$ by setting $B_y(t,n):=-(n\cdot D_y
(\mathbf v_i)(t))\mathbf v_i$
}

\vspace{2ex} Lets verify that this definition is independent from the choice of
$\mathbf v_i$, thus let $\mathbf w_i$ be a different orthonormal basis. Then
they are related with an orthogonal matrix $\mathbf v_i=O_i^j\mathbf w_j$. And
if we compute coordinate change we get
\begin{align*}
    (n\cdot D_y(\mathbf v_i)(t))\mathbf v_i&=(n\cdot D_y(O_i^j\mathbf w_j)(t))\mathbf O_i^k\mathbf w_k
=(n\cdot D_y(O_i^j)(t)\mathbf w_j + O_i^j n\cdot D_y(\mathbf w_j)(t))\mathbf O_i^k\mathbf w_k\\
    &=(O_i^jO_i^k(n\cdot D_y(\mathbf w_j)(t))\mathbf w_k)=(n\cdot D_y(\mathbf w_k)(t))\mathbf w_k
\end{align*}

\vspace{2ex}
Let $\gamma:I\rightarrow M$ be a $\mathcal C^2$ curve. It's tangent is $t=\gamma'$
and its curveture is $\kappa=\gamma''$ and we suppose $|t|=1$. Now lets
differentiat a following expression
\[0=\frac{d}{ds}(t\cdot\mathbf v_i)=\frac{d}{ds}t\cdot\mathbf v_i+t\cdot\frac{d}{ds}\mathbf v_i=\kappa\cdot\mathbf v_i+t\cdot D(v_i)(t)\]
Thus we get coordinates for normal curvature
$\kappa_N = -(t\cdot D(\mathbf v_i)(t))\mathbf v_i=B(t,t)$

\vspace{2ex}
Simillarly if $\phi:U\subset\mathbb R^2\rightarrow M$, then
\[0=\frac{\partial}{\partial x_2}(\frac{\partial}{\partial x_1}\phi\cdot\mathbb{v}_i)=
\frac{\partial^2}{\partial x_2\partial x_1}\phi\cdot\mathbf{v}_i)+\frac{\partial}{\partial x_1}\phi\cdot\frac{\partial}{\partial x_2}\mathbf{v}_i(\phi(x_1,x_2))=
\frac{\partial^2}{\partial x_2\partial x_1}\phi\cdot\mathbf{v}_i)+\frac{\partial}{\partial x_1}\phi\cdot D\mathbf{v}_i(\frac{\partial\phi}{\partial x_2})
\]
And thus $B(\frac{\partial\phi}{\partial x_1},\frac{\partial\phi}{\partial x_2})=(\frac{\partial^2\phi}{\partial x_2\partial x_1})^\bot$
and we remark that $B$ is symmetric.

\vspace{2ex}
\textbf{Definition:} \textit{We define the mean curvature vector $\mathbf H_y$
of $M$ at $y$ to be trace $B_y$; thus}
\[\mathbf H_y=B_y(\mathbf t_i,\mathbf t_i)\]
\textit{where $(\mathbf t_i)$ is an orthonormal basis of $T_yM$.}

\vspace{2ex}
If we rewrite this definition and recall the definition of divergence, we obtain
\[\mathbf H_y=-(t_j\cdot D_{\mathbf t_j}(\mathbf v_i)(\mathbf v_i))\mathbf v_i=\text{div}_M(\mathbf v_i)\mathbf v_i\]
Let $X$ be a vector field, then $X^\bot=(\mathbf v_i\cdot x)\mathbf v_i$ and if we take divergence on $M$, we get
\[\text{div}_M X^\bot=(\mathbf v_i\cdot x)\text{div}_M \mathbf v_i\]
because other components are in orthogonal space. Thus we conclude
\[\text{div}_M X\bot=\mathbf H\cdot X\]
and recalling the integration theorem, we get
\[\int_M \text{div}_MXd\mathcal H^n=\int_{\partial M}\nu\cdot Xd\mathcal H^{n-1}-\int_M \mathbf H\cdot Xd\mathcal H^n\]
