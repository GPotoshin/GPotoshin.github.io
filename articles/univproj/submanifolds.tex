In this section we study once again $\mathbb C^1$ $n$-dimentional submanifold $M$
of $\mathbb R^{n+m}$ to observe several properties of standart constructions to
then motivate a definition for analogus constructions for measures.

\vspace{2ex}
We can define a gradient $\nabla^M$ associated to $M$ by
\[\nabla^M f = (w_i)(Df(w^i))\]
for an orthogonal basis $(w_i)$ of $TM$. This notion is independed from the choice,
since as we have another orthogonal basis $(w_i)=(u_i)P$, where $P$ is orthogonal, 
then $(w^i)=P^t(u^i)$ and thus $(w_i)(Df(w^i))=(u_i)P(Df(P^t(u^i)))=(u_i)PP^tDf(u^i)=(u_i)Df(u^i)$

\vspace{2ex}
If for two vectors $a,b$ by $ab$ we right their scalar product, then we can
introduce diveregence of $X:M\rightarrow\mathbb R^{n+m}$ on $M$ by chosing
a basis $(w_1,\ldots,w_n)$ of $TM$ and setting
\[\text{div}_MX=(w_i)(\nabla^MX^i)\]
where $X=(X^i)_{i\in\llbracket1,n+m\rrbracket}$ are coordinats in orthogonal
extension of $(w_i)$. Or if we associate all indicies to the one basis we get
\[\text{div}_MX=\partial_iX^i\]
Now let suppose that we have curvelinear coordinates $(c^i)$, to them we can
associate a standart basis 
\[\mathbf{c}_i=\mathbf{w}_j\frac{\partial w^j}{\partial c^i}\]
which consists of tangents to constant curbes. Since at from point to point
basis vectors varies in ambient space, we can naturally diffine following coefficients
[for the following part I think that we need to be in $\mathcal C^2$, but apperently new coordinates
are $\mathcal C^1$]
\[\frac{\partial\mathbf{c}_i}{\partial c^j} = \mathbf{c}_k\Gamma_{ji}^k\]
In fact we observe symmetry on botton coeffitiants since
\[\frac{\partial\mathbf{c}_i}{\partial c^j} = \frac{\partial}{\partial c^j}\mathbf{w}_l\frac{\partial w^l}{\partial c^i}
=\mathbf{w}_l\frac{\partial^2 w^l}{\partial c^j\partial c^i}=\mathbf{w}_l\frac{\partial^2 w^l}{\partial c^i\partial c^j}
=\mathbf{c}_k\Gamma_{ij}^k\]
Also in new basis $(\mathbf{c}_k)$ we have scalar product, which is usually denoted by
\[\langle\mathbf{c}_l,\mathbf{c}_k\rangle=g_{lk}=\frac{\partial w^i}{\partial c^l}\frac{\partial w^i}{\partial c^k}\]
which is also symmetric. And if we rewrite a little bit equality for $\Gamma$, we get
\[\mathbf{w}_j\frac{\partial^2 w^j}{\partial c^j\partial c^i}=\mathbf{c}_k\Gamma_{ji}^k
=\mathbf{w}_j\frac{\partial w^j}{\partial c^k}\Gamma_{ji}^k\]
Metric defines an isomorphism $\phi$ from our space $V$ to a dual space $V^*$ by
\[\phi(\mathbf{c}_i):=\langle\mathbf{c}_i,\cdot\rangle=g_{lm}\mathbf{c}^{*l}\otimes\mathbf{c}^{*m}\mathbf{c}_i=g_{li}\mathbf{c}^{*l}\]
This $g_{ij}$ is a matrix of $\phi$ in local coordinates. And a matrix of $\phi^{-1}$ is
$(g_{ij})^{-1}$. Associated to this ismorphism we can construct a dual product by
\begin{align*}
    \langle\cdot,\cdot\rangle^*:&=\phi^{-1}(\langle\cdot,\cdot\rangle)=\phi^{-1}(g_{lm}\mathbf{c}^{*l}\otimes\mathbf{c}^{*m})
=g_{lm}\phi^{-1}(\mathbf{c}^{*l})\otimes\phi^{-1}(\mathbf{c}^{*m})\\
    &=g_{lm}(g^{-1,lr}\mathbf{c}_r)\otimes(g^{-1,mk}\mathbf{c}_k)
=\delta_m^r\mathbf{c}_r\otimes g^{-1,mk}\mathbf{c}_k=g^{-1,rk}\mathbf{c}_r\otimes \mathbf{c}_k
\end{align*}
Or we just usually simplu write $\langle\cdot,\cdot\rangle^*=g^{mk}\mathbf{w}_m\otimes \mathbf{w}_k$.
Let $X'$ be $X$ rewritten in new coordinates, i.e.
\[\mathbf{X}(w^1,...,w^n)=\mathbf{X}'(...v^i(w^1,...,w^n)...)\]
Now we take a partial derivative of both sides:
\begin{align*}
\frac{\partial}{\partial w^j}\mathbf{w}_iX^i&
=\frac{\partial}{\partial w^j}\mathbf{c}_pX'^p
=(\frac{\partial}{\partial c^k}\mathbf{c}_pX'^p)\frac{\partial c^k}{\partial w^j}
=(\mathbf{c}_p\frac{\partial X'^p}{\partial c^k}+\frac{\partial\mathbf{c}_p}{\partial c^k}X'^p)\frac{\partial c^k}{\partial w^j}
=(\mathbf{c}_p\frac{\partial X'^p}{\partial c^k}+\mathbf{c}_l\Gamma^l_{pk}X'^p)\frac{\partial c^k}{\partial w^j}\\
&=(\mathbf{c}_p\frac{\partial X'^p}{\partial c^k}+\mathbf{c}_p\Gamma^p_{lk}X'^l)\frac{\partial c^k}{\partial w^j}
=\mathbf{c}_p(\frac{\partial X'^p}{\partial c^k}+\Gamma^p_{lk}X'^l)\frac{\partial c^k}{\partial w^j}
\end{align*}
Then we can calculate divergence formula by
\begin{align*}
\text{div}\mathbf{X}&=\mathbf{w_j}\cdot\frac{\partial}{\partial w^j}\mathbf{w}_iX^i
=\mathbf{w}_j\cdot\mathbf{c}_p(\frac{\partial X'^p}{\partial c^k}+\Gamma^p_{l,k}X'^l)\frac{\partial c^k}{\partial w^j}
=\mathbf{w}_j\cdot\mathbf{w}_i\frac{\partial w^i}{\partial c^p}(\frac{\partial X'^p}{\partial c^k}+\Gamma^p_{l,k}X'^l)\frac{\partial c^k}{\partial w^j}\\
&=\frac{\partial c^k}{\partial w^j}\frac{\partial w^j}{\partial c^p}(\frac{\partial X'^p}{\partial c^k}+\Gamma^p_{l,k}X'^l)
=\frac{\partial c^k}{\partial c^p}(\frac{\partial X'^p}{\partial c^k}+\Gamma^p_{l,k}X'^l)
=\frac{\partial X'^p}{\partial c^p}+\Gamma^p_{l,p}X'^l
\end{align*}
For the next part we are differentiating $g_{lm}$
\[\frac{\partial}{\partial c^k}g_{lm}=(\frac{\partial}{\partial c^k}\frac{\partial w^i}{\partial c^l})\frac{\partial w^i}{\partial c^m}
+\frac{\partial w^i}{\partial c^l}(\frac{\partial}{\partial c^k}\frac{\partial w^i}{\partial c^m})
=\frac{\partial w^i}{\partial c^r}\Gamma^r_{kl}\frac{\partial w^i}{\partial c^m}+
\frac{\partial w^i}{\partial c^l}\frac{\partial w^i}{\partial c^r}\Gamma^r_{km}
=g_{mr}\Gamma^r_{kl}+g_{lr}\Gamma^r_{km}\]
Now lets consider a following quantity
\[\frac{\partial g_{kl}}{\partial c^m}+\frac{\partial g_{km}}{\partial c^l}-\frac{\partial g_{ml}}{\partial c^k}
=g_{kr}\Gamma^r_{ml}+g_{lr}\Gamma^r_{mk}+g_{kr}\Gamma^r_{lm}+g_{mr}\Gamma^r_{lk}-g_{mr}\Gamma^r_{kl}-g_{lr}\Gamma^r_{km}=2g_{kr}\Gamma^r_{ml}
\]
And if we multiply both sides by $1/2g^{ak}$ we get
\[g^{ak}g_{kr}\Gamma^r_{ml}=\delta_r^a\Gamma^r_{ml}=\Gamma^a_{ml}=\frac{1}{2}g^{ak}(\frac{\partial g_{kl}}{\partial c^m}+\frac{\partial g_{km}}{\partial c^l}-\frac{\partial g_{ml}}{\partial c^k})\]
Now lets take Cristoffel symbol from divergence formula and develop it
\[\Gamma^p_{pl}=\frac{1}{2}g^{pk}(\frac{\partial g_{kl}}{\partial c^p}+\frac{\partial g_{kp}}{\partial c^l}-\frac{\partial g_{pl}}{\partial c^k})
=\frac{1}{2}g^{pk}\frac{\partial g_{kp}}{\partial c^l}\]
Now knowing that $\partial_k(\det A)=\det(A)\text{tr}(A^{-1}\partial_k A)$ and
writing $g=\det(g_{ij})$ we have
\[g^{pk}\partial_l g_{kp}=\text{tr}((g_{kp})^{-1}\partial_l(g_{kp}))=\frac{\partial_l g}{g}\]
And since $\partial_l\sqrt g=\frac{1}{2\sqrt g}\partial_l g$ we have
\[\Gamma^p_{pl}=\frac{1}{\sqrt g}\partial_l\sqrt g\]
and finally we can rewrite divergence formula as
\[\text{div}\mathbf X=\partial_p X'^p+\frac{1}{\sqrt g}\partial_l\sqrt g X'^l
=\frac{1}{\sqrt g}\sqrt g\partial_p X'^p+\frac{1}{\sqrt g}\partial_p\sqrt g X'^p
=\frac{1}{\sqrt g}\partial_p(\sqrt g X'^p)
\]

\vspace{2ex}
\textbf{Theorem:} \textit{Let $M\subset \mathbb R^{n+m}$ be a $n$-dimentional
$\mathcal C^1$ bounded submanifold-with-boundary. Then we have a following result}
\[\int_M \text{div}_MXd\mathcal H^n=\int_{\partial M}X\nu d\mathcal H^{n-1}\;\textit{for every}\;X\in\mathcal C^1(M,TM)\]
\textit{where $\nu$ is an orthogonal unitary vector to the boundry pointing outwards.}

\vspace{1ex}
\textbf{Proof:} since $M$ is compact we can find a finite open cover $U_i$ of
$M$ such that at every $U_i$ we find a local coordinate system. We can associated
a $\mathcal C^\infty$ partition of unit $(\psi_i)$ on $M$ associated to $U_i$ such
that $\psi_i\in\mathcal C^\infty_c(U_i)$ and $\sum\psi_i=1_M$. We shall write
$O_i=U_i\cap M$. And by liniarity of equation we want to prove, we can prove it
jsut for $\psi_iX$, or without loss of generality we will write just $X$. Let
coordinates take values in $V_i$

\vspace{1ex}
If patch $U_i$ is disjoint from $\partial M$, then $V_i$ are open and $X$ is of
a compact support inside $V_i$ and we can integrate by parts
\[0=\int_{V_i}\partial_j 1X^j\sqrt gdc=-\int_{V_i}1\partial_j(X^j\sqrt c)=
-\int_{V_i}\frac{1}{\sqrt g}\partial_j(X^j\sqrt g)\sqrt gdc=-\int_{U_i}\text{div}_M Xd\mathcal H^n\]
Thus inner patches have no contribution. Lets take a patch $U_i$ that intersects
boundary. By definition of submanifold-with-boundary we can introduce coordinates
$c^i$ such that $\{c_n=0\}=\partial M\cap U_i$ and $c_n$ has values only in
$(-\infty,0]$. This time $X$ does not have a compact support inside $V_i$ and
thus in integration by parts we have the 2 terms
\[0=\int_{V_i}\partial_j 1X^j\sqrt gdc=-\int_{U_i}\text{div}_M Xd\mathcal H^n+\int_{\mathbb R^{n-1}}X^n(c',0)\sqrt{g(c',0)}dc'\]
And as we can chose such coordinates, that $X^n(c',0)=\nu\cdot X(c',0)$ we have
\[\int_{U_i}\text{div}_M Xd\mathcal H^n=\int_{\partial M}\nu\cdot Xd\mathcal H^{n-1}\]

\vspace{2ex}
\textbf{Remark:} Result stays true even if do not suppose that $M$ is a
submanifold-with-boundary, as it said on page 43 of "Simon Lectures on
Geometric measure theory", but I do not know a proof. Probably we can alway
approximate a submanifold by a submanifold-with-boubdary and in that case
we have a prove.

\vspace{2ex}
\textbf{Definition:} \textit{Let $\mathbf{v}_i$ be an orthonormal basis of $T_yM$ which
is $C^1$ funciton of $y$. Then we can define \textbf{the second fundamental form}
$B_y:T_yM\times T_yM\rightarrow (T_yM)^\perp$ by setting $B_y(t,n):=-(n\cdot D_y
(\mathbf v_i)(t))\mathbf v_i$
}

\vspace{2ex} Lets verify that this definition is independent from the choice of
$\mathbf v_i$, thus let $\mathbf w_i$ be a different orthonormal basis. Then
they are related with an orthogonal matrix $\mathbf v_i=O_i^j\mathbf w_j$. And
if we compute coordinate change we get
\begin{align*}
    (n\cdot D_y(\mathbf v_i)(t))\mathbf v_i&=(n\cdot D_y(O_i^j\mathbf w_j)(t))\mathbf O_i^k\mathbf w_k
=(n\cdot D_y(O_i^j)(t)\mathbf w_j + O_i^j n\cdot D_y(\mathbf w_j)(t))\mathbf O_i^k\mathbf w_k\\
    &=(O_i^jO_i^k(n\cdot D_y(\mathbf w_j)(t))\mathbf w_k)=(n\cdot D_y(\mathbf w_k)(t))\mathbf w_k
\end{align*}

\vspace{2ex}
\textbf{Definition:}
