In this section we study once again $\mathbb C^1$ $n$-dimentional submanifold $M$
of $\mathbb R^{n+m}$ to observe several properties of standart constructions to
then motivate a definition for analogus constructions for measures.

\vspace{2ex}
We can define a gradient $\nabla^M$ associated to $M$ by
\[\nabla^M f = (w_i)(Df(w^i))\]
for an orthogonal basis $(w_i)$ of $TM$. This notion is independed from the choice,
since as we have another orthogonal basis $(w_i)=(u_i)P$, where $P$ is orthogonal, 
then $(w^i)=P^t(u^i)$ and thus $(w_i)(Df(w^i))=(u_i)P(Df(P^t(u^i)))=(u_i)PP^tDf(u^i)=(u_i)Df(u^i)$

\vspace{2ex}
If for two vectors $a,b$ by $ab$ we right their scalar product, then we can
introduce diveregence of $X:M\rightarrow\mathbb R^{n+m}$ on $M$ by chosing
a basis $(w_1,\ldots,w_n)$ of $TM$ and setting
\[\text{div}_MX=(w_i)(\nabla^MX^i)\]
where $X=(X^i)_{i\in\llbracket1,n+m\rrbracket}$ are coordinats in orthogonal
extension of $(w_i)$. Or if we associate all indicies to the one basis we get
\[\text{div}_MX=\partial_iX^i\]
Now let suppose that we have curvelinear coordinates $(c^i)$, to them we can
associate a standart basis 
\[\mathbf{c}_i=\mathbf{w}_j\frac{\partial w^j}{\partial c^i}\]
which consists of tangents to constant curbes. Since at from point to point
basis vectors varies in ambient space, we can naturally diffine following coefficients
[for the following part I think that we need to be in $\mathcal C^2$, but apperently new coordinates
are $\mathcal C^1$]
\[\frac{\partial\mathbf{c}_i}{\partial c^j} = \mathbf{c}_k\Gamma_{ij}^k\]
In fact we observe symmetry on botton coeffitiants since
\[\frac{\partial\mathbf{c}_i}{\partial c^j} = \frac{\partial}{\partial c^j}\mathbf{w}_j\frac{\partial w^j}{\partial c^i}
=\mathbf{w}_j\frac{\partial^2 w^j}{\partial c^j\partial c^i}=\mathbf{w}_j\frac{\partial^2 w^j}{\partial c^i\partial c^j}
=\mathbf{c}_k\Gamma_{ji}^k\]
Also in new basis $(\mathbf{c}_k)$ we have scalar product, which is usually denoted by
\[\langle\mathbf{c}_l,\mathbf{c}_k\rangle=g_{lk}=\frac{\partial w^i}{\partial c^l}\frac{\partial w^i}{\partial c^k}\]
which is also symmetric. And if we rewrite a little bit equality for $\Gamma$, we get
\[\mathbf{w}_j\frac{\partial^2 w^j}{\partial c^j\partial c^i}=\mathbf{c}_k\Gamma_{ji}^k
=\mathbf{w}_j\frac{\partial w^j}{\partial c^k}\Gamma_{ji}^k\]
Let $X'$ be $X$ rewritten in new coordinates, i.e.
\[\mathbf{X}(w^1,...,w^n)=\mathbf{X}'(...v^i(w^1,...,w^n)...)\]
Now we take a partial derivative of both sides:
\begin{align*}
\frac{\partial}{\partial w^j}\mathbf{w}_iX^i&
=\frac{\partial}{\partial w^j}\mathbf{c}_pX'^p
=(\frac{\partial}{\partial c^k}\mathbf{c}_pX'^p)\frac{\partial c^k}{\partial w^j}
=(\mathbf{c}_p\frac{\partial X'^p}{\partial c^k}+\frac{\partial\mathbf{c}_p}{\partial c^k}X'^p)\frac{\partial c^k}{\partial w^j}
=(\mathbf{c}_p\frac{\partial X'^p}{\partial c^k}+\mathbf{c}_l\Gamma^l_{pk}X'^p)\frac{\partial c^k}{\partial w^j}\\
&=(\mathbf{c}_p\frac{\partial X'^p}{\partial c^k}+\mathbf{c}_p\Gamma^p_{lk}X'^l)\frac{\partial c^k}{\partial w^j}
=\mathbf{c}_p(\frac{\partial X'^p}{\partial c^k}+\Gamma^p_{lk}X'^l)\frac{\partial c^k}{\partial w^j}
\end{align*}
Then we can calculate divergence formula by
\begin{align*}
\text{div}\mathbf{X}&=\mathbf{w_j}\cdot\frac{\partial}{\partial w^j}\mathbf{w}_iX^i
=\mathbf{w}_j\cdot\mathbf{c}_p(\frac{\partial X'^p}{\partial c^k}+\Gamma^p_{l,k}X'^l)\frac{\partial c^k}{\partial w^j}
=\mathbf{w}_j\cdot\mathbf{w}_i\frac{\partial w^i}{\partial c^p}(\frac{\partial X'^p}{\partial c^k}+\Gamma^p_{l,k}X'^l)\frac{\partial c^k}{\partial w^j}\\
&=\frac{\partial c^k}{\partial w^j}\frac{\partial w^j}{\partial c^p}(\frac{\partial X'^p}{\partial c^k}+\Gamma^p_{l,k}X'^l)
=\frac{\partial c^k}{\partial c^p}(\frac{\partial X'^p}{\partial c^k}+\Gamma^p_{l,k}X'^l)
=\sum_p(\frac{\partial X'^p}{\partial c^p}+\Gamma^p_{l,p}X'^l)
\end{align*}
For the next part we are differentiating $g_{lm}$
\[\frac{\partial}{\partial c^k}g_{lm}=\]

\vspace{2ex}
\textbf{Theorem:} \textit{Let $M\subset \mathbb R^{n+m}$ be a $n$-dimentional
$\mathcal C^1$ bounded submanifold-with-boundary. Then we have a following result}
\[\int_M \text{div}_MXd\mathcal H^n=\int_{\partial M}X\nu d\mathcal H^{n-1}\;\textit{for every}\;X\in\mathcal C^1(M,TM)\]
\textit{where $\nu$ is an orthogonal unitary vector to the boundry pointing outwards.}

\vspace{1ex}
\textbf{Proof:} since $M$ is compact we can find a finite open cover $U_i$ of
$M$ such that at every $U_i$ we find a local coordinate system. We can associated
a $\mathcal C^\infty$ partition of unit $(\psi_i)$ on $M$ associated to $U_i$ such
that $\psi_i\in\mathcal C^\infty_c(U_i)$ and $\sum\psi_i=1_M$. And by liniarity
of equation we want to prove, we can prove it jsut for $\psi_iX$, or without
loss of generality we will write just $X$.

\vspace{1ex} If patch $U_i$ is disjoint from $\partial M$, then 

