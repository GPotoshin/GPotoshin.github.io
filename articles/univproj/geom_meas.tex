\documentclass{article}
\usepackage[a4paper,left=3cm,right=3cm,top=1cm,bottom=2cm]{geometry}
\usepackage{amsmath}
\usepackage{amssymb}
\usepackage{hyperref}
\usepackage[english]{babel}

\usepackage{tikz-cd}
\usepackage{array}
\usepackage{graphicx}
\usepackage{mathtools}
\newcommand\mapsfrom{\mathrel{\reflectbox{\ensuremath{\mapsto}}}}
\setlength{\parindent}{0mm}

\usepackage{fontspec}
\setmainfont{Linux Libertine O}
\usepackage{unicode-math}
\setmathfont{Cambria Math}

\begin{document}
\title{
\textit{\small{Georgii Potoshin, 2025}}\\
\vspace{0.3ex}
\textit{\huge{Geometry and measure}}\vspace{1ex}
}
\date{\vspace{-5ex}}
\maketitle


\date{\vspace{-10ex}}

\maketitle

\section{Introduction}
Here are my observations about geometric measure theory.

\subsection{Acknowledged results from measure theory}

% @Checked: 17 mar until Steiner symmetrisation
\subsection{Hausdorff measure}
The Hausdorff measure generalises the notion of measure for lower dimensional
objects in higher-dimensional space. The idea is essentially similar to the
construction of Lebesgue's measure except that we take a lower limit instead of
an infinum. We define a cover of $E$ by sets of diameter less then $\delta$ as
a $\delta$-cover of $E$. And we conceder only countable covers. We note that
\[\mathcal{H}_\delta^s(E)=\inf_{C}\sum_{I\in C}\omega_s\left(\frac{\text{diam}(I)}{2}\right)^s\]
where $s\in\mathbb{R}_{\geq 0}$ is a dimension, $\omega_s\in\mathbb{R}$ is a
coefficient, preferably continuous or smooth as a function of $s$, and $C$ is a
$\delta$-cover of $E$. We may assume that
\[\omega_s = \frac{\pi^{s/2}}{\Gamma(1+s/2)}\]
We define the Hausdorff measure as a limit of the previous value. It exists
because $\mathcal{H}_\delta^s(E)$ is increasing function of $\delta$. We note
\[\mathcal{H}^s(E)=\lim_{\delta\rightarrow0^+}\mathcal{H}^s_\delta(E)\]

I shall introduce the notion of $s$-variation of a cover $S$ as
\[\text{Var}^s(S)=\sum_{I\in S}\omega_s\left(\frac{\text{diam}(I)}{2}\right)^s\]

\textbf{Proposition:} For a natural $n\geq 0$, $\omega_n$ is a volume of a unit
$n$-dimensional ball.
\vspace{1ex}

\subsection{Properties of Hausdorff measure}

\textbf{Proposition:} Hausdorff measure is a Borel measure for regular topology.

\vspace{1ex}
\textbf{Proposition:} In the definition of Hausdorff measure we can consider
only closed or open sets.

\vspace{1ex}
\textbf{Proposition:} Hausdorff measure of dimension $m\in N$ coincide on
$m$-dimensional affine subspaces with their Lebesgue measure.

\vspace{1ex}
\textbf{Proposition:} For a Lipschitz function $f:\mathbb{R}^n\rightarrow
\mathbb{R}^m$ we have the following inequality
\[ \mathcal{H}^s(f[E])\leq\text{Lip}(f)^s\mathcal{H}^s(E) \]
for every $s>0$ and $E\subseteq\mathbb{R}^n$. And $\text{dim}(E)<\text{dim}(f[E])$.

\vspace{1ex}
\textbf{Proposition:} The $n$-dimetional Hausdorff measure traced to a
$n$-dimentional $\mathcal{C}^1$-submanifold of $R^m$ induces the area measure
on this submanifold and coincides with the integral measure via parametrisation
on it.

\vspace{1ex}
\textbf{Remark:} Proofs to those proposition can be found in the book "Geometric
measure theory" be Francesco Maggi.
\subsection{Hausdorff dimension}

To a set $S$ we can associate a number $s=\text{inf}\{a\geq 0\,|\,\mathcal{H}^a
(S)=0\}$. It's called its Hausdorff dimension.

\vspace{1ex}
\textbf{Proposition:}

\section{Dimension of cantor sets}
Here we calculate the dimension of generalized set. Let $n\in\mathbb{N}$ and
$m\in\mathbb{N}^*$ so that $2m<n$. Then we can define $C_k$ ($k\in\mathbb{N}$)
define recursively by agreeing that $C_0=\{[0,1]\}$ and we obtain $C_{k+1}$ from $C_k$
by cutting out the open middle part from each segment of $C_k$ and living side parts
of length $m/n$ of original interval. We will note $C=\lim C_k=\bigcap C_k$.
\[\text{image}\]
Obviously $C_k$ is a $(m/n)^k$-cover of $C$, so
\[ \mathcal{H}_{(m/n)^k}^s \leq \sum_{I\in C_k}\omega_s(\frac{\text{diam}(I)}{2})^s
=\omega_s2^k((m/n)^k)/2)^s=\omega_s/2^s(2(m/n)^s)^k \]
And if $s>\text{log}_{n/m}(2)$ we have right side approaching 0 as $k$ tends to
infinity. That means that $\text{dim}(C)\leq\text{log}_{n/m}(2)$.


Now we need to prove the inequality in the other direction. Let $s=\text{log}_
{n/m}(2)$. And let $S$ be a $(m/n)^k$-cover of $C$. In fact by the construction
$C$ is an intersection of compacts on a real line, so is compact. And by one of
the previous propositions we can conceder only open covers. Then by compactness 
we can leave only a finite number of sets in $S$ and this way we reduce its 
Hausdorff variance and we can extend the resting elements to closed intervals
of the same diameter. This does not change the variance. The new cover is noted
by $S'$. Now in every interval of $S'$ we can find 2 maximal intervals from some
$C_i$ and $C_j$, so the they are disjoint. If we can't do that, then there are no points
of $C$ in this interval and we can throw away that set also. So now we have 2 
maximal intervals $J$ and $J'$ in $I$. They are ordered. Between them we
have an interval $K$ and as they are maximal $I\setminus J\setminus K\setminus J'$ does not contain
any points from $C$ and we can through those parts away from the covering.
By the construction
\[|J|,|J'|\leq \frac{m}{n}\cdot \frac{n}{n-2m}|K|=\frac{m}{n-2m}|K|\]
Now we have $1/2(|J|+|J'|) \leq \frac{m}{n-2m}|K|$
\[|I|^s=(|J|+|J'|+|K|)^s\geq((1+\frac{n-2m}{2m}))(|J|+|J'|))^s=(\frac{n}{m}1/2(|J|+|J'|))^s=2(1/2(|J|+|J'|))^s\geq|J|^s+|J'|^s\]
Where the last step is done by concavity of function $x\mapsto x^s$.
That means that we can reduce this any cover to a $C_k$ cover which has a
smaller $s$-variation. That means that for dimension $s=\text{log}_{n/m}(2)$
the $\mathcal{H}^s(C)$ is finite as the $s$-variation of $C_k$ is always $
\omega_s/2^s$.

\vspace{1ex}
\textbf{Remark:} This is a variation on the proof given in the book "The geometry
of fractal sets" by K. J. Flaconer, generelised to the case of arbitrary $m$ and
$n$. In this book the proof is done for the case $m=1$, $n=3$.

\vspace{1ex}
\textbf{Proposition:} There is a subset of $[0,1]$ with а Hausdorff dimension
$1$, but Lebesgue measure 0.

\vspace{1ex}
To show that we shall use Cantor's sets. Let $C_{m/n}$ be a set discussed in a
previous paragraph. Then $S=\bigcap C_{m/(2m+1)}$ is a set of dimension $1$. As
for every $0\leq s<1$ there is such $m$, that $\text{log}_{n/m}(2)=\text{log}_{
(2m+1)/m}(2)>s$, as $\text{log}_{(2m+1)/m}(2)\rightarrow 1$. And thus $\mathcal
{H}^s(S)>\mathcal{H}^s(C_{m/(2m+1)})=\infty$.

\section{Weak* topology and compactness}
As to a positive measure we can associate an integral, we need to utilise some
results from functional analysis.
\vspace{1ex}

For topological spaces $Y_i$ and a set of functions $f_i:X\rightarrow Y_i$, we can
define the smallest, coarsest topology on $X$ that makes those functions continuous.
By definition such topology is $\tau(\{f_i\})=\bigcap\{\tau\,|\,
\tau$ is a topology on $X$ and $f_i$ are continuous$\}$. As an
example, the product topology is exactly $\tau(\{\pi_i\})$, where $\pi_i$ are
canonical projections.
\vspace{1ex}

\textbf{Proposition:} Let $\tau$ be a topology on $X$. Then $\tau=\tau(\{f_i\})$
if and only if every function $g:W\rightarrow X$ such that $f_i\circ g$ are
continuous is continuous.
\vspace{1ex}

\textbf{Remark:} This is a well-known property of caorsest topology, but I
checked that it's also an alternative characterisation of such topology.

If $\tau=\tau(\{f_i\})$ and $g:W\rightarrow X$ is such function that $f_i\circ
g$ are continuous. It's sufficient to check that for all elements of prebase
of $\tau(\{f_i\})$ the inverse image is open, but the prebase consists of
elements of the form $f_i^{-1}[U]$ and its inverse image is $(f_i\circ g)^{-1}[U]$
which is open by hypotheses.
\vspace{1ex}

If $\tau$ is a such topology, that for every function $g:W\rightarrow X$ it's
continuous if and only if $f_i\circ g$ are continuous, then in particular we
have $\text{id}:(X,\tau)\rightarrow(X,\tau)$ continuous and that means that $f_i = f_i\circ
\text{id}$ are continuous and we have $\tau(\{f_i\})\subseteq\tau$. On the other
hand we have $\text{id}':(X,\tau(\{f_i\}))\rightarrow(X,\tau)$ continuous
because $f_i = f_i\circ\text{id}':(X,\tau(\{f_i\}))\rightarrow Y_i$ are continuous
by the definition of coarsest topology. Thus we have $\text{id}'$ continuous
and that means that $\tau\subseteq\tau(\{f_i\})$. And finally $\tau=\tau(\{f_i\})$.

\vspace{1ex}

\textbf{Theorem (Tychonoff):} Product of compact spaces is compact.
\vspace{1ex}

\textbf{General structure:} Let $I$ be a set of indices and $E_i$ for $i\in I$
be a topological space with a topology $\tau_i$. The prebase of the product
topology on  $\prod_{i\in I} E_i$ is $\{\pi_i^{-1}[U]\,|\,i\in I,U\in\tau_i\}$.
a set of products of open subspaces of one spaces on others. All the finite
intersections form a base of product topology. Its elements are products of
open sets where almost all factors are $E_i$.
\vspace{1ex}

\textbf{Maximal covers:} Let's note that a set of covers that does not contain
finite sub-covers for a partially ordered set with the relation of inclusion.
For every chain we have its union which does not contain a finite sub-cover,
which otherwise would have been in some element of chain. Thus each chain has an
upper bound. By the Zorn's lemma we find a maximal element $M$.
\vspace{1ex}

Let $X$ be a topological space and $M\subseteq\tau$ a maximal cover that does
not contain a finite sub-cover. \textbf{Then if $V\in M^c$, we have $U_1,\ldots,
U_n\in M$ such that $V\cup U_1\cup\ldots\cup U_n=X$.} Because otherwise we
could have added $V$ to $M$ and M would not be maximum. \textbf{If
$U,V\in M^c$ then $U\cap V\in M^c$.} In other words $M^c$ is a multiplicative
system, which is similar to the statement that $\mathfrak{p}^c$ is a multiplicative
for a prime ideal $\mathfrak{p}$. This is true due to the fact that we have
$U_1,\ldots,U_k\in M$ and $V_1,\ldots,V_l\in M$ such that $U\cup U_1\cup\ldots
\cup U_n = X = V\cup V_1\cup\ldots\cup V_l$ and thus $(U\cap V)\cup U_1\cup\ldots
\cup U_k\cup V_1\cup\ldots\cup V_l=X$, which implies that $U\cap V\in M^c$.
\vspace{1ex}

\textbf{Alexander's lemma about prebase: Let $B$ be a prebase of a topological
space $X$. Then if in every cover of $X$ by elements of $B$ there exists a finite
subcover, then the space $X$ is compact.} If $X$ is not compact, then we have
a $M$ maximal cover that does not contain a finite sub-cover. Then to every
$x\in X$ we can associate its neighborhood $V_x\in M$. Then we find some
element of a basis $U_x=U_{1,x}\cup\ldots\cup U_{n_x,x}\subseteq V_x$ where
$U_{i,x}\in B$ are elements of prebase. Thus by maximality $U_x\in M$ as
$U_x\subseteq V_x$. But as $U_x=U_{1,x}\cup\ldots\cup U_{n_x,x}$ and as $M^c$
is a multiplicative system, for some $i$ we have $U_{i,x}\in M$. It means that
in $M$ we have a sub-cover of $X$ by elements of a prebase $B$. And by hypotheses
we can chose a finite sub-cover which gives a contradiction.
\vspace{1ex}

\textbf{Tichonoff theorem's proof:} Let $\mathcal{S}=(U_i)_{i\in I}$ be a cover of a
product $E=\prod_{j\in J} E_j$ of compact space by elements of canonical prebase.
Let's suppose that it does not contain a finite sub-cover. For every $j\in J$
we shall pose $S_j=\{\pi_j^{-1}[V_{i,j}]=U_i\,|\,V_{i,j}\in\tau_j,i\in I_j\}$.
Then $(V_{i,j})_{i\in I}$ cannot be a cover of $E_j$, because otherwise we can
extract a finite sub-cover of $E_j$ and hence of $E$. So we can chose $x_j\in
E_j$ such that $x_j\notin\bigcup_{i\in I_j}V_{i,j}$. Let $x=(x_j)_{j\in J}$ and
it does not lie in every set of $\mathcal{S}$, thus it's not a cover and we get
a contradiction.

\vspace{1ex}
\textbf{Remark:} This is the most non-trivial part of the proof of Banach-Alaoglu
theorem and as I had this proof noted I have decided to also put it here.

\subsection{Topologies on spaces $E$ and $E^*$}
In this section, $E$ is a normed vector space and $E^*$ is its dual space of continuous
1-forms on $E$. On the space $E$, apart from its metric topology, we have
the weak topology $\sigma(E, E^*)=\tau(\{f\}_{f\in E^*})$. As $f\in E^*$ is
continuous with respect to the regular topology, the topology $\sigma(E, E^*)$
is coarser then the regular topology, which we call strong.
\vspace{1ex}

On the space $E^*$, we also have strong topology with the operator norm.
Additionally, we have the weak* topology $\sigma(E^*, E)=\tau(\{v\}_{v\in E})$.

\vspace{1ex}
\textbf{Proposition:} The weak* topology is a trace topology from the space
$\mathbb{R}^E$ with the product topology.

\vspace{1ex}
\textbf{Proof:} Let $\tau(\{\pi_v\}_{v\in E})$ be the trace topology. Then it
is easy to see that $\pi_v=v$ as both function are evaluations at $v$ and thus
$\tau(\{\pi_v\}_{v\in E})=\tau(\{v\}_{v\in E})=\sigma(E^*, E)$ is a weak*
topology.

\vspace{1ex}
\textbf{Remark:} In the book "Functional Analysis" by Haim Brezis, the part
above is done by establishing an homeomorpism and the verification of its bicontinuity.
As you have seen, there is actually nothing substantial to prove since these are 
just two notions of the same concept – projection and evaluation in the dual-space.

\vspace{1ex}
\textbf{Theorem (Banach-Alaoglu):} The closed unit ball $B=\{f\in E^*\,|\,|
f|\leq 1\}$ is compact in the weak* topology $\sigma(E^*, E)$.

\vspace{1ex}
\textbf{Proof:}
\[ B=\left\{f\in\mathbb{R}^E\,|\,
\begin{cases}
    |f(x)|<|x|,\;\forall x\in E\\
    f(\lambda x)=\lambda f(x),\;\forall\lambda\in\mathbb{R}, x\in E\\
    f(x+y)=f(x)+f(y)\;\forall x,y\in E
\end{cases}
\right\} \] 

Hence it is intersection of the following sets $B=K\cap\bigcap_{x,y\in E} A_{x,y}
\cap\bigcap_{x\in E, \lambda\in\mathbb{R}}B_{\lambda,x}$, where $K=\{f\in\mathbb
{R}^E\,|\,|f(x)|\leq|x|\}=\prod_{x\in E}[-|x|, |x|]$ is compact by Tichonoff
theorem, where for $x,y\in E$, we define $A_{x,y}=\{f\in\mathbb{R}^E\,|\,f(x+y)-
f(x)-f(y)=0\}$, which is closed since evaluations and addition are continuous, and
thus $f\mapsto f(x+y)-f(x)-f(y)$ is continuous and $A_{x,y}$. For similar
reasons $B_{\lambda, x}=\{f\in\mathbb{R}^E\,|\,f(\lambda x)-\lambda f(x)=0\}$ is
closed. This proves that $B$ is compact.


\section{Measures and convergence}

\subsection{Vector valued measure}
Let $X$ be a topological space and $V$ a normed vector space, then $\mu:\mathcal{B}(X)
\rightarrow Y$ is a $V$-valued Borel measure if
\[\sum_n\mu(E_n(=\mu(\bigcap_n E_n)\]
for any disjoint countable family $\{E_n\}$ of Borel sets.

\vspace{1ex} Let $\mu$ be a vector valued measure. Then the \emph{total
variation} $\|\mu\|$ of measure $\mu$ is defined by:
\[\|\mu\| = \text{sup}\{\sum_n|\mu(E_n)\,|\,\{E_n\}\}\]


\vspace{1ex}
In the context of geometric measure theory we are interested in the vector
space $E=\mathcal{C}^0_c(\mathbb{R}^n, \mathbb{R}^m)$ with the supremum norm.
Then it's dual space is $E^*=\{L:E\rightarrow\mathbb{R}\,|\,L\text{ is linear
and continious}\}$. Then on the $E^*$ from now and on we will consider the weak*
star topology. To make the connection with measure we shall state the result for
Reisz's representation of $E^*$. In fact every functional $L\in E^*$ can be
represented by vector valued Radon measure $\mu$, such that
\[\langle L,\phi\rangle=\int\phi d\mu\]

\section{Analysis results}

For a ball $B=B(x, r)$ of center $x$ and radius $r$ we shall note $\prescript{\epsilon}{}B
=B(x, (1+\epsilon)r)$ for every $\epsilon>0$.

\vspace{1ex}
\textbf{Vitali's covering theorem:}
Let $\mathcal{F}$ be any collection of nondegenearted closed balls in
$\mathbb{R}^n$ with
\[ \sup\{\text{diam}\,B\,|\, B\in\mathcal{F}\}<\infty \]
Then for every $\epsilon>1$ there exist a countable family $\mathcal{G}$ of
disjoint balls in $\mathcal{F}$ such that
\[\bigcup_{B\in\mathcal{F}}B\subseteq\bigcup_{B\in\mathcal{G}}\prescript{2\epsilon}{}B\]

\vspace{1ex}
\textbf{Proof:}
Set $D=\sup\{\text{diam}\,B\,|\,B\in \mathcal{F}\}$. Set
\[\mathcal{F}_j=\left\{B\in\mathcal{F}\,|\,\frac{D}{\epsilon^j}<\text{diam}\,B\leq\frac{D}{\epsilon^{j-1}}\right\},\quad j=1,2,\ldots\]
We define $\mathcal{G}_j\subseteq\mathcal{F}_j$ as follows
\begin{itemize}
    \item Let $\mathcal{G}_1$ be any maximal disjoint collection of balls in
        $\mathcal{F}_1$.

    \item Assuming $\mathcal{G}_1,\ldots,\mathcal{G}_{k-1}$ have been selected,
        we chose $\mathcal{G}_k$ to be any maximal disjoint subcollection of
        \[ \{B\in\mathcal{F}_k\,|\,B\cap B'=\varnothing\text{ for all }B'\in\bigcup_{j=1}^{k-1}\mathcal{G}_j\}\]
\end{itemize}
They exist by Zorn's Lemma. Finally, define $\mathcal{G}=\bigcup_{j\in\mathbb{N}^*}\mathcal{G}_j$
a collection of disjoint balls and $\mathcal{G}\subseteq\mathcal{F}$.

\vspace{1ex}
Proving that for each ball $B\in\mathcal{F}$, there exists a ball $B'\in\mathcal{G}$
such that $B\cap B'\neq\varnothing$ and $B\subseteq\prescript{\epsilon}{}B'$. Fix
$B\in\mathcal{F}$, there exists and index $j$ such that $B\in\mathcal{F}_j$ and
by maximality of $\mathcal{G}_k$ there exists a ball $B'\in\bigcup_{k=1}^j
\mathcal{G}_k$ with $B\cap B'\neq\varnothing$. But $\text{diam}\,B'>\frac{D}{\epsilon^j}$
and $\text{diam}\,B\leq\frac{D}{\epsilon^{j-1}}$; so that
\[ \text{diam}\,B\leq \frac{D}{\epsilon^{j-1}} < \epsilon\text{diam}\,B'\]
Thus $B\subseteq\prescript{2\epsilon}{}B'$.

\vspace{1ex}
\textbf{Remark:} This is a generalised version of the proof from the book
"Measure theory and fine properties of functions" where it is done for the
smallest integral case $\epsilon = 2$. The generalised proof shows the reason
why the final dilatation is $5 = 1+2\epsilon$, but actually it is true for
dilatation $\ge 3$ and the smallest such integer is 4.

\vspace{1ex}
\textbf{Whitney covering theorem:}
Let $C\subseteq \mathbb{R}^n$ be a closed set and $f:C\rightarrow\mathbb R$,
$d:C\rightarrow\mathbb{R}^{n*}$ be continuous functions. We shall use notions
\[
\begin{aligned}
    &R(y,x)=\frac{f(y)-f(x)-d(x)(y-x)}{|x-y|},\quad\forall x,y\in C, x\neq y \\
    &\rho_k(\delta)=\sup\{|R(x,y)|\, |\, 0<|x-y|\leq\delta, x, y\in K\}
\end{aligned}
\]
if we suppose that for every compact $K\subseteq C$
\begin{equation}
\rho_K(\delta)\rightarrow 0\text{ as }\delta\rightarrow 0
\end{equation}

Then there exists a fuction $\overline f\in\mathcal{C}^1(\mathbb{R}^n,\mathbb{R})$
and $D\overline f|_C=d$.

\vspace{1ex}
\textbf{Remark:}
I seek to give a more explicit version of the proof given in the book "Measure
theory and fine properties of functions". In books that looked at about
geometric measure theory this proof usually is not stated and pointed to the
book of Federer wheres at least in version of that book the theorem is proved
in much more general context and the theorem statement differs from the one we
want.

\vspace{1ex}
\textbf{Proof:}
The main challenge is to find a suitable extension of $f$. To construct this
extension we will select regularly enough points in the complementary set and
make a such function so that on those points it's an extension via averaged linear extrapolation and in
between we interpolate by some close enough points. Let $U=C^c$ be a complementary
open set. Let $r(x)=\frac{1}{4}\min(1,\text{dist}(x,C))$.
By Vitali's covering theorem there exist a countable set
$\{x_j\}_{j\in\mathbb{N}}$ and a countable set of disjoint closed balls $\{B_j=B(x_j, r(x_j))\}_
{j\in\mathbb{N}}$ such that $\bigcup_{j\in\mathbb{N}}\prescript{2}{}B_j=U$. We
need $\frac{1}{2}$ in the definition of $r(x)$ to make sure that $\prescript{2}{}
B_j\subseteq U$. Then for every $x\in U$ we shall define $S_x=\{x_j\,|\,B(x,2r(x))
\cap B(x_j,2r(x_j))\neq\varnothing\}$. 

\vspace{1ex}
Now we chech that $S_x$ is bounded for each dimention.
Let $x_j\in S_x$ then $|r(x)-r(x_j)| \leq 1/4|x-x_j|$ because $|r(x)-r(x_j)|=
1/4|\min(1,\text{dist}(x,C))-\min(1,\text{dist}(x_j,C))|$ and without loss of
generality we can consider 3 cases:
\begin{enumerate}
    \item $\text{dist}(x,C),\text{dist}(x_j,C)>1$ then $|\min(1,\text{dist}(x,C))
        -\min(1,\text{dist}(x_j,C))|=0\leq|x-x_j|$.
    \item $\text{dist}(x,C)\leq1,\text{dist}(x_j,C)>1$, then $|\min(1,\text{dist}
        (x,C))-\min(1,\text{dist}(x_j,C))| = 1-\text{dist}(x,C)<\text{dist}(x_j,C)
        -\text{dist}(x,C)=|x_j-s|-|x-s|\leq|x_j-x|$, where $s$ is a projection
        of $x$ on $C$. 
    \item $\text{dist}(x,C)\leq\text{dist}(x_j,C)\leq1$, then $|\min(1,\text{dist}
        (x,C))-\min(1,\text{dist}(x_j,C))| = \text{dist}(x_j,C)-\text{dist}(x,C)
        \leq|x_j-x|$.
\end{enumerate}

So we have $|r(x)-r(x_j)| \leq 1/4|x-x_j|\leq1/4|2r(x)-2r(x_j)|=1/2(r(x)+r(x_j))$
as $x_j\in S_x$. And hence
\[
\begin{aligned}
    &r(x)-r(x_j)\leq1/2(r(x)+r(x_j))\;\Rightarrow\;r(x)\leq 3r(x_j)\\
    &r(x_j)-r(x)\leq1/2(r(x)+r(x_j))\;\Rightarrow\;r(x_j)\leq 3r(x)
\end{aligned}
\]
In addition we have $|x-x_j|+r(x_j)\leq2(r(x)+r(x_j))+r(x_j)\leq2r(x)+6r(x)+3r(x)=
11r(x)$. Which means that $B(x_j, r(x_j))\subseteq B(x, 11r(x))$ and since
$B(x_j, r(x_j))$ are disjoint we have an inquality on volumes:
\[
    \#S_x\omega_n(r(x)/3)^n\leq\#S_x\omega_n(r(x_j))^n=\sum_{x_j\in S_x}\text{Vol}
    \,B_j\leq\text{Vol}(B(x,11r(x))=\omega_n(11r(x))^n
\]
Therefor $\#S_x\leq(3\cdot11)^n=33^n$ is bounded by a fixed constant in each
dimention.

\vspace{1ex}
The goal of that part is to construct the function $\overline f$. Let $\mu:
\mathbb{R}\rightarrow\mathbb{R}$ be a $\mathcal{C}^\infty$ function such that
$0\leq\mu\leq1$, $\mu(t) = 1$ if $t\leq 1$ and $\mu(t)=0$ if $t\geq 2$. Then
for each $j=1,\ldots$ we set $u_j(x)=\mu\left(\frac{|x-x_j|}{2r(x_j)}\right)$
for $x\in\mathbb{R}^n$. Then $u_j\in\mathcal{C}^\infty$, $0\leq u_j\leq1$ and
$u_j\equiv 1$ on $B(x_j, 2r(x_j))$ and $u_j\equiv 0$ on $B(x_j, 4r(x_j))$. 

\section{Countably n-rectifiable sets}


\section{Grassmannian}

In this section we introduce the topological space $G(m,n)$.
\vspace{1ex}

Similarly to projective spaces $P\mathbb{R}^n$ one can generalise this notion to
smaller subspaces than hyperplanes. The set of $m$ dimensional subspaces of a
vector space $\mathbb{R}^n$ is called grassmannian and noted by $G(m,n)$. It
has a topology identified from a topology of orthogonal projection on
$m$-dimensional subspaces.

\section{Varifold}

An $m$-dimensional varifold $V$ is a Radon measure over $\mathbb{R}^n\times
G(n,m)$ endowed with a product topology. We say $\|V\|$ is a measure in
$\mathbb{R}^n$ which is reciprocally projection of a varifold $V$ by $\pi_1^{-1}$.

\textbf{Proposition: For varifolds we concider weak* topology. Then we have a
convergence criteria that $V_i\rightarrow V$ if and only if
\[\int fdV_i\rightarrow\int fdV\]
for every continuous function $f:\mathbb{R}^n\times G(m,n)\rightarrow R$ with a
compact support.}
\vspace{1ex}

\end{document}
