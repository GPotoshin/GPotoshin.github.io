\textbf{Convention:} To any product we associate morphisms denoted by $\pi_i$,
where $i$ is an index of product component we are projecting to or is the component
itself. For example for $A\times B$ we have $\pi_A:A\times B\rightarrow A$, for
$A^n$ we have $\pi_i:A^n\rightarrow A$ with $i\in\llbracket1,n\rrbracket$, for $\prod_{i\in I} A_i$
we have $\pi_{i_0}:\prod_{i\in I} A_i\rightarrow A_{i_0}$ for $i_0\in I$.
Similarly we introduce a notion $i$ for morphisms associated to coproducts.
Notions of morphisms $\pi$ and $i$ are reserved only for those needs.

\vspace{2ex}
\textbf{Definition:} \textit{An \textbf{outer measure} on $X$ is a set function on $X$ with
values in $[0,\infty]$ with
\begin{itemize}
    \item $\mu(\varnothing)=0$
    \item $E\subseteq\bigcup_{h\in\mathbb{N}}E_h\quad\Rightarrow\quad\mu(E)\leq\sum_{h\in\mathbb{N}}\mu(E_h)$
\end{itemize}
}

\vspace{2ex}
\textbf{Carathéodory's theorem:} \textit{If $\mu$ is an outer measure on $X$ and $\mathcal
M(\mu)$ is the family of those $E\subseteq X$ such that
\[\mu(F)=\mu(E\cap F)+\mu(F\setminus E),\quad\quad\forall F\subseteq X\]
then $\mathcal M(\mu)$ is a $\sigma$-algebra and $\mu$ is a measure on $\mathcal
M(\mu)$.}

\vspace{2ex}
\textbf{Definition:} \textit{$\mu$ is a \textbf{Borel measure} on a topological space $X$
if it is an outer measure on $X$ such that $\mathcal B(X)\subseteq\mathcal M(\mu)$.}

\vspace{2ex}
\textbf{Definition:} \textit{A measure $\mu$ is said to be \textbf{absolutely continuous
with respect to} measure $\lambda$ if for any set $A$, $\lambda(A)=0$
implies $\mu(A)=0$ and we write it $\mu << \lambda$.}

\vspace{2ex}
\textbf{Definition:} \textit{We say that a Borel measure $\mu$ is \textbf{regular} if
for every $F\subseteq X$ there exists a Borel set $E\in\mathcal B(X)$ such that
\[F\subseteq E,\quad\quad\quad\quad\quad\quad\mu(F)=\mu(E)\]}

\vspace{2ex}
\textbf{Definition:} \textit{An outer measure $\mu$ on $X$ is \textbf{locally finite} if
$\mu(K)<\infty$ for every compact set $K\subseteq X$.}

\vspace{2ex}
\textbf{Definition:} \textit{An outer measure $\mu$ is a \textbf{Radon measure} on a
topological space if it is locally finite, Borel regular measure on $X$.}

\vspace{2ex}
\textbf{Property of Radon measures on $\mathbb{R}^n$:} \textit{If $\mu$ is a Radon
measure on $\mathbb{R}^n$, then
\[
    \mu(E)=\inf\{\mu(A)\,|\,E\subseteq A,\,A\textnormal{ is open}\}
          =\sup\{\mu{K}\,|\,K\subseteq E,\,K\textnormal{ is compact}\}
\]}

\vspace{2ex}
\textbf{Definition:} \textit{For a function $f:X\rightarrow Y$ between metric spaces we can
define its \textbf{Lipschitz constant} $\text{Lip}(f) = \textnormal{inf}\{L\in
\mathbb{R}\,|\,d(f(x),f(y))\leq Ld(x,y) \forall x,y\in X\}$}

\vspace{2ex}
\textbf{Definition:} \textit{A Borel measure $\mu$ on a metric space $X$ is
said to be a \textbf{doubling measure} if there exists a constant $C$ such that
\[0<\mu(B(x,2r))\leq C\mu(B(x,r))<\infty\]}
\subsection{Hausdorff measure}
The Hausdorff measure generalises the notion of measure for lower dimensional
objects in higher-dimensional space or even in an arbitrary metric space. The
idea is essentially similar to the
construction of Lebesgue's measure except that we take a lower limit instead of
an infinum. We define a cover of $E$ by sets of diameter less then $\delta$ as
a $\delta$-cover of $E$. And we conceder only countable covers. We note that
\[\mathcal{H}_\delta^s(E)=\inf_{C}\sum_{I\in C}\omega_s\left(\frac{\text{diam}(I)}{2}\right)^s\]
where $s\in\mathbb{R}_{\geq 0}$ is a dimension, $\omega_s\in\mathbb{R}$ is a
coefficient, preferably continuous or smooth as a function of $s$, and $C$ is a
$\delta$-cover of $E$. We may assume that
\[\omega_s = \frac{\pi^{s/2}}{\Gamma(1+s/2)}\]
We define the Hausdorff measure as a limit of the previous value. It exists
because $\mathcal{H}_\delta^s(E)$ is increasing function of $\delta$. We note
\[\mathcal{H}^s(E)=\lim_{\delta\rightarrow0^+}\mathcal{H}^s_\delta(E)\]

I shall introduce the notion of $s$-variation of a cover $S$ as
\[\text{Var}^s(S)=\sum_{I\in S}\omega_s\left(\frac{\text{diam}(I)}{2}\right)^s\]

\textbf{Proposition:} \textbf{For a natural $n\geq 0$, $\omega_n$ is a volume of a unit
$n$-dimensional ball.}
\vspace{1ex}

\subsection{Properties of Hausdorff measure}

\textbf{Proposition:} \textit{Hausdorff measure is a Borel measure for regular topology.}

\vspace{2ex}
\textbf{Proposition:} $\mathcal H^0$ is the counting measure.

\vspace{2ex}
\textbf{Proposition:} \textit{In the definition of Hausdorff measure we can consider
only closed or open sets.}

\vspace{2ex}
\textbf{Proposition:} \textit{Hausdorff measure of dimension $m\in N$ coincide on
$m$-dimensional affine subspaces with their Lebesgue measure.}

\vspace{2ex}
\textbf{Proposition:} \textit{The $n$-dimetional Hausdorff measure traced to a
$n$-dimentional $\mathcal{C}^1$-submanifold of $R^m$ induces the area measure
on this submanifold and coincides with the integral measure via parametrisation
on it.}

\vspace{1ex}
\textbf{Remark:} Proofs to those proposition can be found in the book "Geometric
measure theory" be Francesco Maggi \cite{maggi}.
\subsection{Hausdorff dimension}

To a set $S$ we can associate a number $s=\text{inf}\{a\geq 0\,|\,\mathcal{H}^a
(S)=0\}$. It's called its Hausdorff dimension.

\vspace{2ex}
\textbf{Proposition:} \textit{If $E\subseteq\mathbb R$ then $\dim(E)\in [0, n]$.
Moreover $\mathcal H^s(E)=\infty$ for every $s<\dim(E)$ and $\mathcal H^S(E)\in
(0,\infty)$ implies $s=\dim(E)$.}

\vspace{2ex}
\textbf{Proposition:} \textit{If $A$ is an open set in $\mathbb R^n$, then $\dim
(A) = n$.}

\vspace{2ex}
\textbf{Proposition:} \textit{For a Lipschitz function $f:\mathbb{R}^n\rightarrow
\mathbb{R}^m$ we have the following inequality
\[ \mathcal{H}^s(f[E])\leq\text{Lip}(f)^s\mathcal{H}^s(E) \]
for every $s>0$ and $E\subseteq\mathbb{R}^n$ and $\text{dim}(E)<\text{dim}(f[E])$.}

