Let's suppose that we have $\Sigma$ an $n$-dimensional $\mathcal C^1$ submanifold of
$\mathbb R^{n+m}$ and an open set $U$. Let's fix a $\mathcal C^2$ map $\phi:(-1,1)
\times U\rightarrow U$ and a closed subset $K\subset U$ such that:
\begin{enumerate}
    \item $\phi_t:=\phi(t,\cdot)$ is a $\mathcal C^1$-diffeomorphism for every $t$
    \item $\phi_0=\text{id}$
    \item $\phi_t|_{K^c}=\text{id}$ for every $t$
\end{enumerate}
Thus $\phi$ is a controlled geometrical variation of $K$. Let $V:=\partial_t\phi
(0,\cdot)$ be a velocity vector field and $A:=\partial_t^2\phi(0,\cdot)$ an
acceleration field, then
\[\phi(t,\cdot)=\text{id}+tV+\frac{t^2}{2}A+o(t^2)\]

Lets note $M:=\phi(\cdot,\Sigma\cap K)$ and $\psi_t:=\phi_t|_{\Sigma}$

\vspace{1ex}
\textbf{Definition:} By first and second variations of families $M$ we mean 
following derivatives $\mathcal H^n(M)'(0)$ and $\mathcal H^n(M)''(0)$.

\vspace{1ex}
If we recall the area formula then we can compute variations as a function of
$\phi$
\[\mathcal H^n(M(t))=\int_{M(0)}(\text{J}\psi_t)d\mathcal H^n\]
[Introduce earlier this formula]
Thus we construct a map
\[D_x\psi_t:T_x\Sigma\rightarrow\mathbb R^{n+m}=\text{id}+tD_xV+\frac{t^2}{2}D_xA+o(t^2)\]
[does $D_xA$ even exist in this context?] Lets chose an orthogonal basis $\tau$
for $T_xM$ and an orthogonal basis $e$ for $\mathbb R^{n+m}$. Then in those
basis we can write matrix of $D_x\psi_t$.
\[[D_x\psi_t]_i^l=\tau_i^l+tD_xV^l(\tau_i)+\frac{t^2}{2}D_xA^l(\tau_i)+o(t^2)\]
To calculate $\text{J}\psi_t$ we need to know $(D_x\psi_t)^*\circ(D_x\psi_t)$
\begin{align*}
[&(D_x\psi_t)^*\circ(D_x\psi_t)]_i^j=\sum_l[D_x\psi_t]_j^l[D_x\psi_t]_i^l=\sum_l
(\tau_j^l+tD_xV^l(\tau_j)+\frac{t^2}{2}D_xA^l(\tau_j)+o(t^2))\\
&(\tau_i^l+tD_xV^l(\tau_i)+\frac{t^2}{2}D_xA^l(\tau_i)+o(t^2))\\
&=\tau_j\cdot\tau_i+t(\tau_j\cdot D_xV(\tau_i)+\tau_i\cdot D_xV(\tau_j))+\tau^2
(\frac{1}{2}(\tau_j\cdot D_xA(\tau_i)+\tau_i\cdot D_xA(\tau_j))+D_xV(\tau_i)\cdot D_xV(\tau_j))+o(t^2)
\end{align*}
