Let's suppose that we have $\Sigma$ an $n$-dimensional $\mathcal C^1$ submanifold of
$\mathbb R^{n+m}$ and an open set $U$. Let's fix a $\mathcal C^2$ map $\phi:(-1,1)
\times U\rightarrow U$ and a closed subset $K\subset U$ such that:
\begin{enumerate}
    \item $\phi_t:=\phi(t,\cdot)$ is a $\mathcal C^1$-diffeomorphism for every $t$
    \item $\phi_0=\text{id}$
    \item $\phi_t|_{K^c}=\text{id}$ for every $t$
\end{enumerate}
Thus $\phi$ is a controlled geometrical variation of $K$. Let $V:=\partial_t\phi
(0,\cdot)$ be a velocity vector field and $A:=\partial_t^2\phi(0,\cdot)$ an
acceleration field, then
\[\phi(t,\cdot)=\text{id}+tV+\frac{t^2}{2}A+o(t^2)\]

Lets note $M:=\phi(\cdot,\Sigma\cap K)$ and $\psi_t:=\phi_t|_{\Sigma}$

\vspace{1ex}
\textbf{Definition:} By first and second variations of families $M$ we mean 
following derivatives $\mathcal H^n(M)'(0)$ and $\mathcal H^n(M)''(0)$.

\vspace{1ex}
If we recall the area formula then we can compute variations as a function of
$\phi$
\[\mathcal H^n(M(t))=\int_{M(0)}(\text{J}\psi_t)d\mathcal H^n\]
Thus we construct a map
\[D_x\psi_t:T_x\Sigma\rightarrow\mathbb R^{n+m}=\text{id}+tD_xV+\frac{t^2}{2}D_xA+o(t^2)\]
Lets chose an orthogonal basis $\tau$
for $T_xM$ and an orthogonal basis $e$ for $\mathbb R^{n+m}$. Then in those
basis we can write matrix of $D_x\psi_t$.
\[[D_x\psi_t]_i^l=\tau_i^l+tD_xV^l(\tau_i)+\frac{t^2}{2}D_xA^l(\tau_i)+o(t^2)\]
To calculate $\text{J}\psi_t$ we need to know $(D_x\psi_t)^*\circ(D_x\psi_t)$
\begin{align*}
&[(D_x\psi_t)^*\circ(D_x\psi_t)]_i^j=\sum_l[D_x\psi_t]_j^l[D_x\psi_t]_i^l=\sum_l
(\tau_j^l+tD_xV^l(\tau_j)+\frac{t^2}{2}D_xA^l(\tau_j)+o(t^2))\\
&(\tau_i^l+tD_xV^l(\tau_i)+\frac{t^2}{2}D_xA^l(\tau_i)+o(t^2))\\
&=\tau_j\cdot\tau_i+t(\tau_j\cdot D_xV(\tau_i)+\tau_i\cdot D_xV(\tau_j))+t^2
(\frac{1}{2}(\tau_j\cdot D_xA(\tau_i)+\tau_i\cdot D_xA(\tau_j))+D_xV(\tau_i)\cdot D_xV(\tau_j))+o(t^2)
\end{align*}

This can be abstracted as $I+tS+t^2T$ and we want to calculate a development
of determinant of it. We have a Jacobi formula $\det(N(t))'=\det(N(t))\text{tr}
(N(t)^{-1}N'(t))$ and it's derivative
\[\det(N(t))''=\det(N(t))'\text{tr}(N(t)^{-1}N'(t))+\det(N(t))\text{tr}(-N(t)^
{-1}N'(t)N(t)^{-1}N'(t)+N(t)^{-1}N''(t))\]
and is we replace $N(t)=I+tS+t^2T$, then $\det(I+tS+t^2T)'=\det(I+tS+t^2T)\text{tr}
((I+tS+t^2T)^{-1}(S+2tT))$ and at zero we have a following equality
$\det(N(t))'|_{t=0}=\text{tr}(S)$ and for a second derivative we have
\[\det(N(t))''|_{t=0}=\text{tr}(S)^2-\text{tr}(S^2)+2\text{tr}(T)\]
Thus we can write a Taylor Polynomial
\[\det(I+tS+t^2T)=1+t\text{tr}(S)+\frac{t^2}{2}(2\text{tr}(T)-\text{tr}(S^2)+\text{tr}(S)^2)+o(t^2)\]
In our case we are calculating $\det((D_x\psi_t)^*\circ(D_x\psi_t))=(\text{J}(\psi|_t)(x))^2$
\[
    \text{tr}(S)=\sum_i(\tau_i\cdot D_xV(\tau_i)+\tau_i\cdot D_xV(\tau_i))=2\text{div}_\Sigma V(x)\\
\]
\[
    \text{tr}(T)=\text{div}_\Sigma A+\sum_i|DV(\tau_i)|^2
\]
\[
    \text{tr}(S^2)=(\tau_i\cdot D_xV(\tau_j)+\tau_j\cdot D_xV(\tau_i))(\tau_i\cdot D_xV(\tau_j)+\tau_j\cdot D_xV(\tau_i))
    =2(\tau_i\cdot D_xV(\tau_j)(\tau_j\cdot D_xV(\tau_i))
\]
And for Taylor Polynomial of a root we have
\[\sqrt{1+x}=1+\frac{1}{2}x-\frac{1}{8}x^2+o(x^2)\]
And composed Taylor Polynomial will be
\[\sqrt{1+(at+bt^2)}=1+\frac{1}{2}at+\frac{t^2}{2}(b-\frac{1}{4}a^2)+o(t^2)\]
In our case the value for $a$ is
\[a=2\text{div}_\Sigma V(x)\]
And thus we can express the first variation
\[\det(M(t))'|_{t=0}=\text{div}_\Sigma V(x)\]
For $b$ the expression is a little longer
\[b=2(\text{div}_\Sigma V(x))^2+(\text{div}_\Sigma A+\sum_i|DV(\tau_i)|^2)-(\tau_i\cdot D_xV(\tau_j)(\tau_j\cdot D_xV(\tau_i))\]
And we have an expression for the second variation
\[\det(M(t))''|_{t=0}=(\text{div}_\Sigma V(x))^2+(\text{div}_\Sigma A+\sum_i|DV(\tau_i)|^2)-(\tau_i\cdot D_xV(\tau_j)(\tau_j\cdot D_xV(\tau_i))\]
