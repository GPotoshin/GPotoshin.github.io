An $m$-dimensional varifold $V$ is a Radon measure over $\mathbb{R}^n\times
G(n,m)$ endowed with a product topology. We say $\|V\|$ is a measure in
$\mathbb{R}^n$ that is reciprocal projection of a varifold $V$ by $\pi_1^{-1}$.

\vspace{2ex}
\textbf{Proposition:} \textit{For varifolds we consider weak-$*$ topology. Then we have a
convergence criteria that $V_i\rightarrow V$ if and only if
\[\int fdV_i\rightarrow\int fdV\]
for every continuous function $f:\mathbb{R}^n\times G(m,n)\rightarrow R$ with a
compact support.}

\vspace{1ex}
\textbf{Remark:} Varifolds allow us separate the notion of a tangent plane from
geometric properties (a similar idea is actually used for normal maps in computer
graphics). They allow us to consider possibly many planes at the same point with
different masses and also to change the mass of geometric figures.

\vspace{2ex}
Thus, if we have a surface $M$, we can consider varifolds of the kind $\mathcal H^n
\llcorner M\otimes T$, where $T$ is a Radon measure on the Grassmannian. Or, if
we want to be even more precise, we can have a tangent function $T$ that provides
a tangent at a given point and a mass function $\theta$. Then, we consider a
varifold of type $f\rightarrow \int f(x,T(x))\theta(x)d\mathcal H^n$. And we
are free to choose tangents.

\vspace{2ex}
Sometimes, a choice for a tangent space can be made naturally. Then, for
example, to an $n$-rectifiable set $E$, we can naturally associate
a varifold $v(E,\theta)$ by introducing the following functional
\[\langle V_E\,|\, f\rangle:=\int_Ef(x,T(E,x))\theta(x)d\mathcal H^m,\quad f\in\mathcal C_c(\mathbb R^n, G(m,n))\]
if $\theta=1$ we note such varifold as $v(E)$, if $\theta\in\mathbb{Z}$,
we say that varifold is \emph{integral}.

\vspace{2ex}
\textbf{Definition:} \textit{We say that a varifold $V$ has a tangent space $T$
with multiplicity $\theta\in(0,\infty)$, using a similar idea as for limit
planes. That is to say, if}
\[V_{x,r}\rightharpoonup \theta v(T)\]
\textit{Where $V_{x,r}$ is a varifold passed through a dilation mapping at $x$
with renormalisation coefficient, i.e.}
\[V_{x,r}(A):=r^{-n}V(\{(ry+x,S)\;|\;(y,S)\in A\})\]

\vspace{2ex}
\textbf{Definition:} By analogy to the variation of submanifolds, one can define
first variation of varifolds as a following functional
\[\delta V:X\in\mathcal C^1_c(\mathbb R^n,\mathbb R^n)\rightarrow\int_{\mathbb R^n\times G_{d,n}}\text{div}_SXdV(x,s)\]

\vspace{2ex}
If it happens that first variation is continuous, which is true for rectifiable
varifolds. We can decompose first variation by Riesz's representation theorem and
Radon-Nykodim theorem into two parts, with respect to mass measure as

\[\delta V = -\mathbf Hd\|V\|+\delta V^s\]

Where $\delta V^s$ is a singular part of the decomposition, which is just a restriction
of the first variation on the set where Radon-Nykodym derivative explodes. This
part corresponds to the integration on the boundary for manifolds. Here, $\mathbf H$
is a generalized mean curvature, and in the case when the varifold is actually
a rectifiable set, it is precisely its mean curvature because of the uniqueness of
the decomposition.

\subsection{Examples}
Let's consider an arc of radius 1 of center $(0,0)$ starting at $(1,0)$
and ending at $(-1,0)$ with a mass $\theta$ and call it $A$ as a varifold, then
\[\text{div}_AX=\partial_{\mathbf e_\alpha}(X^\alpha\mathbf e_\alpha+X^r\mathbf e_r)\cdot e_r
=\partial_{\mathbf e_\alpha}X^\alpha+X^r\]
Then, the result for a first variation is
\begin{align*}
\delta A(X)&=\int_A(\partial_{\mathbf e_\alpha}X^\alpha+X^r)\theta d\mathcal H^1
=\int_A\partial_{\mathbf e_\alpha}X^\alpha\theta d\mathcal H^1+\int_A X^rd\mathcal H^1
=[X^\alpha\theta]_0^\pi-\int_AX^\alpha\partial_{\mathbf e_\alpha}\theta d\mathcal H^1+\int_A X^rd\mathcal H^1\\
&=[X^\alpha\theta]_0^\pi-\int_AX\cdot(\partial_{\mathbf e_\alpha}\theta\mathbf e_\alpha-\theta\mathbf e_r)d\mathcal H^1
\end{align*}
Thus, a singular term is $[X^\alpha\theta]_0^\pi$ which corresponds to integration
on the boundary with a correction for a new mass $\theta$. Then, the curvature vector
is $\mathbf H=\partial_{\mathbf e_\alpha}\theta\mathbf e_\alpha-\theta\mathbf e_r$,
which deviates from a standard one by scaling in the orthogonal direction by mass
$\theta$ and by a gradient of $\theta$ in the tangent direction.

\vspace{2ex}
Let's consider a segment $S=[a,b]$ of a horisontal line in $\mathbb R^2$, and lets construct a
varifold of $S$ which tangent plane is orthogonal to $S$ at every point. Then,
\[\delta V(X) = \int_S\text{div}_{S^\bot}Xd\mathcal H^1 = \int_S(\partial_2 X)\cdot e_2d\mathcal H^1=
\int_S(\partial_2 X)d\mathcal H^1\cdot e_2=\partial_2\int_S X^2d\mathcal H^1\]
And we get a perpendicular direction on the boundary. We can also remark that in
this case the functional $\delta V$ is no longer continuous for a supremum norm on
the function space. We can have a sequence of functions that uniformly converges
to 0, but perpendicular derivative is positive and uniformly converges to the
infinity.
