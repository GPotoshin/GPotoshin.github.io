\documentclass{article}
\usepackage[a4paper,left=3cm,right=3cm,top=1.5cm,bottom=2cm]{geometry}
\usepackage{amsmath}
\usepackage{amssymb}
\usepackage{hyperref}
\usepackage[english]{babel}

\usepackage{tikz-cd}
\usepackage{array}
\usepackage{graphicx}
\usepackage{mathtools}
\newcommand\mapsfrom{\mathrel{\reflectbox{\ensuremath{\mapsto}}}}
\setlength{\parindent}{0mm}
\usepackage{stmaryrd}

\usepackage{fontspec}
\setmainfont{Linux Libertine O}
\usepackage{unicode-math}
\setmathfont{Cambria Math}

\usepackage[nottoc]{tocbibind}

\begin{document}
\title{
\vspace{-1cm}
\textit{\small{Georgii Potoshin, 2025}}\\
\vspace{0.3ex}
\textit{\huge{Geometries et measures pour l'ENS }}\vspace{1ex}
}
\date{\vspace{-5ex}}
\maketitle

\section{Introduction}
Aujourd'hui, je vais vous présenter une partie de mon projet de recherche
universitaire de cette année. Ces derniers mois, j'ai étudié la théorie
géométrique de la mesure, en me basant principalement sur les \textit{"Lecture
Notes"} du Professeur Giovanni Alberti et l'ouvrage \textit{"Sets of Finite
Perimeter and Geometric Variational Problems"} de Francesco Maggi. J'aimerais
tout d'abord vous exposer les domaines d'intérêt de cette théorie.

\vspace{1ex}
L'analyse classique utilise généralement les fonctions pour étudier les figures.
Mais cette méthode peut parfois être assez restrictive, car les fonctions sont
généralement supposées être de classe $\mathcal C^1$, et cela entraîne une
restriction sur les types de figures que l'on peut étudier. De plus, chaque
fois que l'on utilise des fonctions, cela introduit une paramétrisation, ce qui
ajoute encore de la structure. Même si l'on peut l'utiliser pour la
classification des structures lisses, cela nous éloigne (ou 'nous décentre') de
notre point de concentration, car c'est un peu moins lié à la géométrie
(intrinsèque).

\vspace{1ex}
Dans la théorie géométrique de la mesure, l'une des idées est d'utiliser les
mesures à la place des fonctions, c'est-à-dire que les mesures vont caractériser
les objets géométriques, et plus généralement quelles opérations avec les
mesures peut-on faire pour les étudier.
Par ailleurs, l'un des avantages de travailler avec les mesures est que l'on
peut introduire une convergence des formes qui apparaît naturellement,
permettant de traiter des limites de séquences d'objets qui peuvent changer de
topologie, développer des singularités ou même dégénérer (par exemple, une
suite de surfaces se réduisant à une courbe). Alors que pour les méthodes
classiques, à mon avis, il y a plus de difficultés à le faire, pour les raisons
déjà discutées, liées au paramétrage.

\vspace{1ex}
Ici, je vais vous présenter quelques aspects qui illustrent cela.
\section{Brièvement sur la mesure de Hausdorff}
Comme le but de cette théorie est de fournir un instrument plus universel, on a
toujours le besoin de calculer l'aire. D'une part, la façon dont on calcule
l'aire détermine notre surface, et nous avons donc besoin d'une généralisation
de la mesure sur les variétés. Il existe plusieurs candidats pour une telle
mesure, et leur comparaison a constitué un travail majeur au début du XXe
siècle. Je vais utiliser le candidat le plus couramment employé: la mesure de
Hausdorff. On l'obtient par une construction très similaire à celle de la
mesure de Lebesgue, car elle utilise la construction de Carathéodory. Plus
précisément, dans

\[\Psi_\delta(E) := \inf\left\{\sum_{n\in\mathbb N}\rho(F_n)\;|\;\{F_n\}_{n\in
\mathbb N}\subseteq\mathcal F, \text{diam}(F_n)<\delta,
E\subseteq\bigcup_{n\in\mathbb N}F_n\right\} \]

Plus précisément, on prend $\mathcal F=\mathcal P(X)$ où $X$ est l'espace
métrique, et $\rho(F)=\omega_s(\text{diam}(F)/2)^s$, où $\omega_s$ est le volume
de la boule unité en dimension $s$. Dans ce cas, on note $\mathcal H_\delta^s:=
\Psi_\delta$. L'étape restante est de prendre la limite sur $\delta$. Ainsi, on
pose $\mathcal H^s := \lim_{\delta \rightarrow 0^+} \mathcal H^s_\delta$.

\subsection{Propriétés utiles}
\begin{itemize}
    \item La mesure de Hausdorff est une mesure de Borel.
    \item La mesure de Hausdorff de dimension $m \in \mathbb{N}$ coïncide sur
        les sous-espaces affines de dimension $m$ avec leur mesure de Lebesgue.
    \item La mesure de Hausdorff de dimension $n$ restreinte à une sous-variété
        $\mathcal{C}^1$ de $\mathbb{R}^m$ de dimension $n$ induit la mesure
        d'aire sur cette sous-variété et coïncide avec la mesure intégrale
        obtenue par paramétrisation.
    \item La mesure de Hausdorff de dimension strictement supérieure à la
        dimension de l'ensemble est nulle.
    \item La mesure $\mathcal H^n$ restreinte à un ensemble localement $\mathcal
        H^n$-fini est une mesure de Radon.
    \item La mesure $\mathcal{H}^n$ est invariante par les mouvements, et si
        $h$ est une homothétie de coefficient $k$, alors $\mathcal{H}^n(h[E])=
        |k|^n \mathcal{H}^n(E)$.
\end{itemize}

\subsection{Densité supérieure de la mesure de Hausdorff}
\textbf{Définition:} \textit{Soit $E$ un sous-ensemble de Borel d'un espace
métrique $X$. La densité supérieure de dimension-$n$ (par rapport à $\mathcal H 
^n$) de $E$ au point $x$ est définie par
\[\Theta^{*}_n(E, x) := \limsup_{r \to 0^+} \frac{\mathcal{H}^n(E \cap B(x,r))}{\omega_nr^n}\]
}

\textbf{Proposition:} \textit{Soit $E$ un sous-ensemble de Borel d'un espace
métrique $\mathbb R^{n+m}$, et supposons que $E$ est $\mathcal H^n$-localement
fini. Alors les propriétés suivantes sont vraies
\begin{itemize}
    \item $\Theta^∗_n(E,x)=0$, pour $\mathcal H^n$-presque tout $x\in E^c$
    \item $\Theta^∗_n(E,x)\leq 1$, pour $\mathcal H^n$-presque tout $x\in E$
\end{itemize}
}

\section{Convergence des mesures}
Les mesures permettent non seulement de calculer les intégrales, mais on peut
aussi les utiliser pour modéliser les figures geometrique et pour tester 
différentes proprietes de ces figure.
L'une des idées fondamentales au cœur de la théorie géométrique de la mesure
est que l'on peut remplacer les figures (ou 'objets géométriques') par les
mesures induites sur ces figures.
\[E\rightsquigarrow \mu\,\llcorner\hspace{-1mm}E\]
Maitenent nous avons besoins de compare 2 mesure, est pour cela on peut comparer
les valeurs des integrales des fonctions sur ces mesure, autrement dit on allons
traite les mesure comme les fonctionnelle lineaire. De plus si deux figures
sont proche l'un à l'autre, alors on veut que leurs measure associer donnes
les valeure assez proche l'un à l'autre, autrement dit on vue que les valeurs
des fonction rest borné dans les petite voisinage et qu'il ne vari pas trop.
Donc on ne traite que les fonction continue. Dernierement, comme nous voudrons
traiter des figures possiblement non borné, nous emrons que l'evolution des mesure
sur les integales soit bein definit et fini. Donc on ne traite que les fonction
continue a support compacte $\mathcal C_c(X)$.

\vspace{1ex}
Cela nous permet d'avoir une notion de
convergence des formes, équivalente à une convergence des mesures. Nous en
verrons des exemples plus tard, mais maintenant, j'aimerais préciser le type de
convergence que nous utiliserons.

\vspace{1ex}
En effet, cette définition est aussi justifiée par des résultats remarquables:

\vspace{2ex}
\textbf{Théorème de représentation de Riesz des mesures à valeurs dans $\mathbb
R^d$:} \textit{Dans $\mathbb{R}^n$, les fonctionnelles linéaires sur $\mathcal
C_0(\mathbb R^n, \mathbb R^d)$ sont précisément les mesures à valeurs dans
$\mathbb R^d$. Alors ces mesures forment un espace dual à $\mathcal C_c(\mathbb
R^n, \mathbb R^d)$}

\vspace{2ex}
\textbf{Théorème de représentation de Riesz des mesures de Radon:} \textit{Dans
$\mathbb R^n$, les fonctionnelles linéaires positives sur $\mathcal C_c(
\mathbb R^n)$ sont précisément les mesures de Radon.}

\vspace{2ex}
Dans ce contexte, les coordonnées d'une mesure sont les fonctions, et la
convergence que l'on considère est la convergence par coordonnées. Lorsque l'on
considère une paire d'espaces $(E^∗,E)$, la topologie induite par les
évaluations sur les éléments de $E$ est appelée topologie faible*. Comme la
topologie d'évaluation en coordonnées est une topologie produit, la boule unité
est incluse dans un produit d'espaces compacts, et est donc compacte. Ce
résultat est appelé \textit{le théorème de Banach-Alaoglu}. Autrement dit, si
nous avons une suite de mesures bornée, alors elle admet une sous-suite
convergente.

\vspace{1ex}
Le deuxième théorème (de représentation de Riesz) signifie, en partie, qu'une
limite de mesures de Radon, si elle existe, est une mesure de Radon.

\vspace{2ex}
\textbf{Proposition:} \textit{Soit $(\mu_n)_{n\in\mathbb N}$ une suite de
mesures positives de Radon, et supposons qu'elle converge vers $μ$,
alors
\begin{enumerate}
    \item Pour tous ensemble ouvert $K\subseteq\mathbb R^n$ nous avons $\liminf_{n
        \rightarrow\infty}\mu_n(A)\geq\mu(A)$.
    \item Pour tous ensemble compact $K\subseteq\mathbb R^n$ nous avons $\limsup_{n
        \rightarrow\infty}\mu_n(K)\leq\mu(K)$.
    \item Pour tous ensemble $E$ relativement compact, tel que $\mu(\partial
        E)=0$ nous avons $\lim_{n\rightarrow\infty}\mu_n(E)=\mu(E)$.
\end{enumerate}}

\vspace{1ex}
\textbf{Démonstration:}
Nous allons démontrer simultanément les propositions 1 et 2. Soit $K \subset A$,
où $K$ est compact et $A$ est ouvert. Considérons une fonction $f\in\mathcal{C}
_c(X)$ telle que $\chi_K\leq f\leq\chi_A$. Pour une mesure de Radon $\nu$, nous
avons alors:
\[\nu(K) \leq \int f \,d\nu \leq \nu(A)\]
Et en considérant les limites, nous obtenons:
\[\limsup \mu_i(K) \leq \limsup \int f \,d\mu_i = \int f \,d\mu \leq \mu(A)\]
\[\mu(K) \leq \int f \,d\mu = \liminf \int f \,d\mu_i \leq \liminf \mu_i(A)\]
Comme nous traitons des mesures de Radon et que ces inégalités sont vraies pour
tout compact $K$ et tout ouvert $A$, nous pouvons passer à la limite. Les
lignes se transforment alors en :
\[\limsup \mu_i(K) \leq \mu(K)\]
\[\mu(A) \leq \liminf \mu_i(A)\]
Le point trois est le conseconce de 2 points précedents. En effet nous avouns
\[\limsup\mu_i(\overline E)\leq\mu(\overline E)=\mu(\r E)\leq\liminf\mu_i(\r E)\]

\vspace{2ex}
\textbf{Remarque:}  La troisième proposition nous donne la propriété que les
mesures d'une suite de figures convergent vers l'autre figure; ainsi,
localement dans la boule, l'aire converge aussi.

\section{Espaces Tangentes}
\textbf{Définition de cône:} \textit{Soit $\alpha$ un angle fixé, soit $x\in\mathbb
R^{n+m}$ un point et soit $V$ un plan de dimension $n$ dans $\mathbb R^{n+m}$.
Le \textbf{cône d'angle $\alpha$ autour de $V$ centré en $x$} est défini par:
\[\mathcal C(x,V,\alpha)=\{x'\in\mathbb R^{n+m}\;|\; |x′−x|\sin(\alpha)\geq d(x−x′, V)\}\]
}

\vspace{2ex}
\textbf{Définition:} \textit{Soit $V\in G(n+m,n)$ un plan de dimension $n$. Si
$E$ est un ensemble de Borel et $x\in E$ un point, alors $V$ est un \textbf{plan
tangent fort} à $E$ en $x$ si et seulement si pour tout $\alpha >0$ il existe un rayon $r_0
>0$ tel que
\[E∩B(x, r_0)\subseteq C(x, V, \alpha)\]}

\vspace{1ex}
\textbf{Remarque:} Évidemment, si un tel plan existe, il est unique. Ce plan
est défini purement par un aspect géométrique.

\vspace{2ex}
Maintenant, nous allons introduire l'aspect lié à la mesure. Nous allons
considérer les cas où il y a des points à l'extérieur du cône, mais leur
densité décroît plus rapidement que leur concentration sur le plan. De plus,
les points doivent se concentrer sur tous les côtés du point x. Ce n'est donc
pas une notion plus faible.

\vspace{1ex}
\textbf{Definition:} \textit{Soit $V\in G(n+m,n)$ un plan de dimension $n$. Si 
$E$ est un ensemble de Borel et $x\in E$ un point, alors $V$ est un plan tangent
approximatif à $E$ en $x$ si et seulement si pour tout $\alpha>0$, on a
\[\mathcal H^n((E∩B(x,r))\setminus\mathcal C(x, V, \alpha)) = o(n^d)\]
et
\[\mathcal H^n((E∩B(x, r))∩\mathcal C(x, V, α))\sim \omega_nr^n\]
}

\vspace{2ex}
Finalement, on peut définir un espace tangent à l'aide de la convergence faible*.
Les espaces satisfaisant cette définition sont généralement aussi appelés
approximatifs, mais pour réduire la confusion, ici je vais les appeler les
espaces limites.

\vspace{1ex}
\textbf{Définition:} 
\textit{Soit $\psi_{x,r}:\mathbb R^{n+m}\rightarrow \mathbb R^{n+m}=x'\mapsto
\frac{x'-x}{r}$ l'application de dilatation. Et soit $E_{x,r}$ l'image de $E$
par $\psi_{x,r}$.  Un plane $V$ de dimension $n$ est un plan limite à
l'ensemble $E$ au point $x$ si et seulement si
\[\mathcal H^n\,\llcorner\hspace{-1mm}E_{x,r}\rightharpoonup\mathcal H^n\,\llcorner\hspace{-1mm}V\] 
}

\vspace{2ex}
\textbf{Proposition:} \textit{Un plan limite est un plan approximatif.}

\vspace{1ex}
\textbf{Démonstration:} Soit $V$ un plan limite de $E$ en $x$. Notons $\mu:=
\mathcal{H}^n\,\llcorner V$ et $\mu_r:=\mathcal{H}^n\,\llcorner E_{x,r}$. Comme
$\mu$ et $\mu_r$ sont des mesures de Radon et que $B(0,1)$ est relativement
compact et que sa frontière est $\mu$-négligeable, alors $\mu_r(B(0,1))
\rightarrow_{r\rightarrow 0} \mu(B(0,1)) = \omega_n$.
De plus, nous avons
\[\mu_{x,r}(B(0,1)) = \mathcal{H}^n(\psi_{x,r}[E\cap B(x, r)]) = \frac{1}{r^n}
\mathcal{H}^n(E\cap B(x, r))\]
et donc
\[\mathcal{H}^n(E\cap B(x, r)) \sim \omega_n r^n\]
Si, dans les constructions précédentes, on remplace $B(0,1)$ par $B(0,1)\cap
\mathcal{C}(0,V,\alpha)$, on trouve
\[\mathcal{H}^n(E\cap B(x, r)\cap\mathcal{C}(x,V,\alpha)) \sim \omega_n r^n\]
Et si l'on prend la différence de ces deux égalités en les divisant par $r^n$,
on trouve que
\[\frac{\mathcal{H}^n(E\cap B(x, r)) - \mathcal{H}^n(E\cap B(x, r)\cap\mathcal{C}(x,V,\alpha))}{r^n} = \frac{\mathcal{H}^n(E\cap B(x, r)\setminus\mathcal{C}(x,V,\alpha))}{r^n} \sim 0\]
Et donc le plan est approximatif.

\vspace{2ex}
\textbf{Proposition:} \textit{Soient $\Sigma, \Sigma' \subseteq \mathbb{R}^{n+m}$ des surfaces de dimension $n$ de classe $C^1$. Alors les plans tangents sont égaux $\mathcal{H}^n$-presque partout pour les points $x \in \Sigma \cap \Sigma'$.}

\vspace{1ex}
Pour le prouver, nous prenons un point $x\in\Sigma\cap\Sigma'$ tel que $T_x\Sigma\neq T_x \Sigma'$. Alors, localement en $x$, les surfaces sont représentées par des submersions $F,G:\mathbb R^{m+n}\rightarrow\mathbb R^m$, c'est-à-dire $\Sigma\cap A=F^{-1}(0)\cap A$ et $\Sigma'\cap B=G^{-1}(0)\cap B$, où $A$ et $B$ sont des voisinages ouverts de $x$.

\vspace{1ex}
Introduisons une nouvelle fonction $(F, G):\mathbb R^{m+n}\rightarrow\mathbb R^{2m}$ définie par $x\mapsto (F(x),G(x))$. Le différentiel de $(F,G)$ est une matrice de 2 blocs, l'un au-dessus de l'autre. Ils sont placés verticalement parce qu'en fait, la paire $(F,G)$ est une colonne et nous avons $D(F,G)=(DF,DG)^t$. Alors $A\cap B\cap\Sigma\cap\Sigma'=(F,G)^{-1}(0)$ et nous avons une représentation de l'intersection. Remarquons que $(F, G)$ n'est pas nécessairement une submersion. Prenons dans le différentiel de $(F,G)$ les indices $(i_k)_{k\in\llbracket1,M\rrbracket}$ d'un ensemble maximalement linéairement indépendant de lignes. Son cardinal est au moins $m$ car les lignes du différentiel de $F$ sont indépendantes, et il est strictement plus grand, car sinon les espaces tangents en $x$ coïncideraient.
\[D\left(\begin{array}{cc} F\\ G\end{array}\right) =
    \left(\begin{array}{cc} \vdots \\DF_i\\ \vdots\\ DG_k\\ \vdots\end{array}\right)\]
où $F_i=\pi_i\circ F$ et $G_k=\pi_k\circ G$ sont des fonctions coordonnées. Alors, si nous ne conservons que ces lignes dans $(F,G)$, nous aurons une submersion H
\[H=\left(\left(\begin{array}{cc} F\\ G\end{array}\right)_j\right)_{j\in(i_k)}\]
Ainsi, nous avons $H:\mathbb R^{n+m}\rightarrow\mathbb R^{M}$, où $m<M<n+m$ est le rang de $H$ en $x$. Par conséquent, nous obtenons une surface de dimension $n+m-M<n$, $H^{-1}(0)\cap A\cap B=\Sigma''$ et $\Sigma\cap\Sigma'\cap A\cap B\subset\Sigma''$, car $(F,G)(z)=0\Rightarrow H(z)=0$. Ainsi $\Sigma\cap\Sigma'\cap A\cap B$ a une mesure $\mathcal H^n$ nulle.

\vspace{1ex}
Finalement, nous avons montré que l'ensemble cible $S=\{x\in\Sigma\cap\Sigma'\;|\;T_x\Sigma \neq T_x\Sigma'\}$ autour de chaque point a une boule ouverte où sa mesure est nulle. Puisque de toute couverture ouverte nous pouvons extraire une sous-couverture dénombrable (car notre espace est séparable), nous avons prouvé que l'ensemble entier est $\mathcal H^n$-nul.
\section{Ensembles rectifiables}
Nous allons considérer une généralisation des surfaces. Un ensemble rectifiable
est effectivement construit à partir d'un nombre dénombrable de morceaux de
surfaces. Le nom 'rectifiable' indique que de tels ensembles posséderont des
propriétés utiles pour les espaces tangents.

\vspace{2ex}
\textbf{Définition:}
Un ensemble $E \subseteq \mathbb{R}^{n+m}$ est $n$-rectifiable s'il est décomposé
en parties
\[E=\bigcup_{i\in\mathbb N}E_i\]
et si $E_0$ est $\mathcal{H}^n$-nul, et si l'une des propositions équivalentes est vraie:
\begin{enumerate}
    \item pour $i\in\mathbb N^*$ nous avons $E_i\subseteq \Sigma_i$ et $\Sigma_i$ est une sous-variété $\mathcal C^1$ de dimension $n$.
    \item pour $i\in\mathbb N^*$ nous avons $E_i\subseteq \textnormal{im}\,F_i$ et $F_i\in\mathcal C^1(\mathbb R^n,\mathbb R^{n+m})$.
    \item pour $i\in\mathbb N^*$ nous avons $E_i\subseteq \textnormal{im}\,F_i$ et $F_i\in\textnormal{Lip}(\mathbb R^n,\mathbb R^{n+m})$.
\end{enumerate}

\vspace{2ex}
\textbf{Proposition:} \textit{Un ensemble de Borel $d$-rectifiable $E \subseteq
\mathbb{R}^n$ admet une unique application $T$ (à un ensemble $\mathcal
H^d$-négligeable près) de $E$ vers la variété grassmannienne $G(n, d)$ telle
que pour chaque surface $\Sigma$ de dimension $d$ de classe $\mathcal{C}^1$, on
a $T_x\Sigma = T(x)$ pour $\mathcal{H}^d$-presque tout $x \in \Sigma \cap E$.
Un tel $T$ est appelé fibré tangent faible.}

\vspace{1ex}
\textbf{Démonstration:} Nous avons $E \subseteq E_0 \cup \bigcup \Sigma_i$,
d'où nous pouvons définir un fibré comme suit. Pour $x \in E_0$, nous pouvons
prendre ce que nous voulons ; pour $x \in \Sigma_1$, nous prenons $T_x\Sigma_1$;
et pour $x \in \Sigma_s \setminus \bigcup_{i=1}^{s-1} \Sigma_i$, nous prenons
$T_x\Sigma_s$. C'est une condition nécessaire car les plans doivent être
$\mathcal{H}^d$-presque partout égaux aux plans tangents à ces surfaces. La
condition pour un fibré tangent faible est satisfaite grâce à la proposition
précédente.

\vspace{2ex}
\textbf{Théorème:} \textit{Si $E$ est un ensemble de Borel, $n$-rectifiable,
$\mathcal{H}^n$-localement fini, alors le fibré tangent faible $T(x)$ est le
plan tangent limite à $E$ en $x$ pour $\mathcal{H}^d$-presque tout $x\in E$.}

\vspace{1ex}
\textbf{Démonstration:}
Nous allons montrer que $T_x\Sigma_i$ est un \textbf{plan tangent limite} à $E$
en $x$ pour $\mathcal H^n$-presque tout $x\in E\cap\Sigma_i$. Associé à ce
plan, nous considérons quatre mesures: $\mu_{x,r}:=\mathcal H^n\,\llcorner
E_{x,r}$, $\nu_{x,r}:=\mathcal H^n\,\llcorner\Sigma_{i,x,r}$, $\eta_{x,r}:=
\mathcal H^n\,\llcorner (\Sigma_i\setminus E)_{x,r}$ et $\sigma_{x,r}:=\mathcal
H^n\,\llcorner (E\setminus\Sigma_i)_{x,r}$. On remarque alors que
$\mu_{x,r}=\nu_{x,r}-\eta_{x,r}+\sigma_{x,r}$.

\vspace{1ex}
Dans $x$, la surface $\Sigma_i$ est localement représentée par une immersion
$\phi: T_x\Sigma_i \cap U \rightarrow \Sigma_i \cap V$. Nous pouvons supposer
que $D\phi(0)=\text{Id}$ et que $B(0,1)\subseteq U,V$. Soit $f\in\mathcal
C_c$, sans perdre de généralité, nous pouvons supposer que $\text{spt}(f)
\subseteq B(0,1)$. Donc, $\psi_{x,r}\circ \phi(h)=(h+o(h))/r$. Si l'on ne prend que les $h<r$, on
trouve que $\phi_{x,r}=\psi_{x,r}\circ \phi|_{B(0,r)}\circ \psi_{0,1/r}:B(0,1)
\rightarrow \Sigma_{i,x,r}$ est donnée par $h\mapsto (rh+|rh|\epsilon(rh))/r=h+|h|
\epsilon(rh)$.
De plus, le différentiel $D\phi_{x,r}$ converge vers l'identité:
\[D\phi_{x,r}=rD\phi|_{B(0,r)}1/r=D\phi|_{B(0,r)}\rightarrow \text{Id}\]
Par conséquent, l'intégrale converge:
\[\int fd\nu_{x,r}=\int_{\Sigma_i\cap B(x,r)}f(s)d\mathcal H^n(s)=\int_{B(0,1)}
f(\phi_{x,r}(s)) J\phi_{x,r}(s)ds\rightarrow_{r\rightarrow 0}\int_{T_x\Sigma_i}
f(s)ds\]
Ainsi, nous avons la convergence faible des mesures:
\[\nu_{x,r}\rightharpoonup\mathcal H^n\,\llcorner T_x\Sigma_i\]

\vspace{1ex}
En suite nous remarquons, que $\lambda_r\rightharpoonup 0\Leftrightarrow
\lambda_r(B_R)\rightarrow 0$ pour tous rayons $R$.

\vspace{1ex}
Pour les mesures $\eta_{x,r}$ et $\sigma_{x,r}$, il suffit de traiter le cas
$B(0,1)$, parce qu'on dilate les figures de toute façon. Pour $\eta_{x,r}$,
nous avons:
\[\eta_{x,r}(B(0, 1))=\mathcal H^n(B(0, 1)\cap(\Sigma_{i,x,r}\setminus E_{x,r}))=
\frac{1}{r^n}\mathcal H^n(B(x,r)\cap(\Sigma_i\setminus E))\rightarrow 0\]
Ceci est vrai pour presque tout $x$, par la première propriété de la densité
supérieure de la mesure de Hausdorff, car $x \notin \Sigma_i \setminus E$.

\vspace{1ex}
Finalement, pour $\sigma_{x,r}$, nous observons que:
\[\mu_{x,r}(B(0,1))=\nu_{x,r}(B(0,1))-\eta_{x,r}(B(0,1))+\sigma_{x,r}(B(0,1))\]
En passant à la limite, nous obtenons:
\[\lim_{r\to 0}\mu_{x,r}(B(0,1))=\omega_n-0+\lim_{r\to 0}\sigma_{x,r}(B(0,1))\]
Et comme, par la deuxième propriété de densité, $\limsup_{r\to 0}\mu_{x,r}(B(0,1))
\le \omega_n$ presque partout, nous trouvons que $\lim_{r\to 0}\sigma_{x,r}(B(0,1))=0$.

Ainsi, nous avons la convergence faible:
\[\mu_{x,r}=\nu_{x,r}-\eta_{x,r}+\sigma_{x,r}\rightharpoonup\mathcal H^n\,\llcorner T(x)-0+0\]
pour presque tout $x$.

\vspace{1ex}
\textbf{Remarque:} Dans cette démonstration, il faut être un peu plus prudent
avec les domaines des fonctions, mais ce n'est normalement qu'une question
technique.

\vspace{10ex}
\textbf{Find the Galois group $\text{Gal}(\mathbb{Q}(\sqrt{2} + \sqrt{3})/\mathbb{Q})$ and describe all intermediate subfields.}
\end{document}
