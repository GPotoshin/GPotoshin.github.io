Measures not only allow us to compute integrals, but they can also be used to
model geometric figures and to test different properties of these figures. One
of the fundamental ideas at the heart of geometric measure theory is that one
can replace figures with the measures induced on these figures.
\[E\rightsquigarrow \mu\,\llcorner\hspace{-1mm}E\]
Now, we need to compare two figures. To do this, we can compare the values of
integrals of functions with respect to associated measures; in other words, we will
treat measures as linear functionals. Furthermore, if two figures are close to
one another, then we want their associated measures to yield sufficiently close
values. This implies that we want the function values to remain bounded in small
neighborhoods and not vary too much. Therefore, we only consider continuous
functions. Lastly, since we want to be able to work with possibly unbounded
figures, we would like the integration of measures with respect to functions to
be well-defined and finite. Thus, we only use continuous functions with compact
support, $\mathcal C_c(X)$.

\vspace{1ex}
This allows us to establish a notion of convergence of shapes, equivalent to a
convergence of measures. We'll see examples of this later, but for now, I'd
like to specify the type of convergence we'll be using.

We are treating measures as linear functionals on the space $\mathcal C_c(X)$.
In this context, convergence is defined by the behavior of these functionals on
each test function; specifically, we are considering that the integral values
converge for every function in $\mathcal C_c$. Such convergence is called 
weak-* convergence. Before discussing this convergence further, I shall first
demonstrate that there is a space of measures which serves as the dual space to
$\mathcal C_c$.

\subsection{Topologies on spaces $E$ and $E^*$}
For topological spaces $Y_i$ and a set of functions $f_i:X\rightarrow Y_i$, we can
define the smallest, coarsest topology on $X$ that makes those functions continuous.
By definition such topology is $\tau(\{f_i\})=\bigcap\{\tau\,|\,
\tau$ is a topology on $X$ and $f_i$ are continuous$\}$. As an
example, the product topology is exactly $\tau(\{\pi_i\})$, where $\pi_i$ are
canonical projections.
\vspace{1ex}

\textbf{Proposition:} \textit{Let $\tau$ be a topology on $X$. Then $\tau=\tau(\{f_i\})$
if and only if every function $g:W\rightarrow X$ such that $f_i\circ g$ are
continuous is continuous.}
\vspace{1ex}

\textbf{Remark:} This is a well-known property of caorsest topology, but I
checked that it is also an alternative characterisation of such topology.

If $\tau=\tau(\{f_i\})$ and $g:W\rightarrow X$ is such function that $f_i\circ
g$ are continuous. It's sufficient to check that for all elements of prebase
of $\tau(\{f_i\})$ the inverse image is open, but the prebase consists of
elements of the form $f_i^{-1}[U]$ and its inverse image is $(f_i\circ g)^{-1}[U]$
which is open by hypotheses.
\vspace{1ex}

If $\tau$ is a such topology, that for every function $g:W\rightarrow X$ it is
continuous if and only if $f_i\circ g$ are continuous, then in particular we
have $\text{id}:(X,\tau)\rightarrow(X,\tau)$ continuous and that means that $f_i = f_i\circ
\text{id}$ are continuous and we have $\tau(\{f_i\})\subseteq\tau$. On the other
hand we have $\text{id}':(X,\tau(\{f_i\}))\rightarrow(X,\tau)$ continuous
because $f_i = f_i\circ\text{id}':(X,\tau(\{f_i\}))\rightarrow Y_i$ are continuous
by the definition of coarsest topology. Thus we have $\text{id}'$ continuous
and that means that $\tau\subseteq\tau(\{f_i\})$. And finally $\tau=\tau(\{f_i\})$.

\vspace{1ex}

\textbf{Tichonoff's Theorem:} \textit{Product of compact spaces is compact.}
\vspace{1ex}

\textbf{General structure:} Let $I$ be a set of indices and $E_i$ for $i\in I$
be a topological space with a topology $\tau_i$. The prebase of the product
topology on  $\prod_{i\in I} E_i$ is $\{\pi_i^{-1}[U]\,|\,i\in I,U\in\tau_i\}$.
a set of products of open subspaces of one spaces on others. All the finite
intersections form a base of product topology. Its elements are products of
open sets where almost all factors are $E_i$.
\vspace{1ex}

\textbf{Maximal covers:} Let's note that a set of covers that does not contain
finite sub-covers for a partially ordered set with the relation of inclusion.
For every chain we have its union which does not contain a finite sub-cover,
which otherwise would have been in some element of chain. Thus each chain has an
upper bound. By the Zorn's lemma we find a maximal element $M$.
\vspace{1ex}

Let $X$ be a topological space and $M\subseteq\tau$ a maximal cover that does
not contain a finite sub-cover. \textbf{Then if $V\in M^c$, we have $U_1,\ldots,
U_n\in M$ such that $V\cup U_1\cup\ldots\cup U_n=X$.} Because otherwise we
could have added $V$ to $M$ and M would not be maximum. \textbf{If
$U,V\in M^c$ then $U\cap V\in M^c$.} In other words $M^c$ is a multiplicative
system, which is similar to the statement that $\mathfrak{p}^c$ is multiplicative
for a prime ideal $\mathfrak{p}$. This is true due to the fact that we have
$U_1,\ldots,U_k\in M$ and $V_1,\ldots,V_l\in M$ such that $U\cup U_1\cup\ldots
\cup U_n = X = V\cup V_1\cup\ldots\cup V_l$ and thus $(U\cap V)\cup U_1\cup\ldots
\cup U_k\cup V_1\cup\ldots\cup V_l=X$, which implies that $U\cap V\in M^c$.
\vspace{1ex}

\textbf{Alexander's lemma about prebase: Let $B$ be a prebase of a topological
space $X$. Then if in every cover of $X$ by elements of $B$ there exists a finite
subcover, then the space $X$ is compact.} If $X$ is not compact, then we have
a $M$ maximal cover that does not contain a finite sub-cover. Then to every
$x\in X$ we can associate its neighborhood $V_x\in M$. Then we find some
element of a basis $U_x=U_{1,x}\cup\ldots\cup U_{n_x,x}\subseteq V_x$ where
$U_{i,x}\in B$ are elements of prebase. Thus by maximality $U_x\in M$ as
$U_x\subseteq V_x$. But as $U_x=U_{1,x}\cup\ldots\cup U_{n_x,x}$ and as $M^c$
is a multiplicative system, for some $i$ we have $U_{i,x}\in M$. It means that
in $M$ we have a sub-cover of $X$ by elements of a prebase $B$. And by hypotheses
we can chose a finite sub-cover which gives a contradiction.
\vspace{1ex}

\textbf{Tichonoff theorem's proof:} Let $\mathcal{S}=(U_i)_{i\in I}$ be a cover of a
product $E=\prod_{j\in J} E_j$ of compact space by elements of canonical prebase.
Let's suppose that it does not contain a finite sub-cover. For every $j\in J$
we shall pose $S_j=\{\pi_j^{-1}[V_{i,j}]=U_i\,|\,V_{i,j}\in\tau_j,i\in I_j\}$.
Then $(V_{i,j})_{i\in I}$ cannot be a cover of $E_j$, because otherwise we can
extract a finite sub-cover of $E_j$ and hence of $E$. So we can chose $x_j\in
E_j$ such that $x_j\notin\bigcup_{i\in I_j}V_{i,j}$. Let $x=(x_j)_{j\in J}$ and
it does not lie in every set of $\mathcal{S}$, thus it is not a cover and we get
a contradiction.

\vspace{1ex}
\textbf{Remark:} This is the most non-trivial part of the proof of Banach-Alaoglu
theorem and as I had this proof noted I have decided to also put it here.

\vspace{1ex}
In this section, $E$ is a normed vector space and $E^*$ is its dual space of continuous
1-forms on $E$. On the space $E$, apart from its metric topology, we have
the weak topology $\sigma(E, E^*)=\tau(\{f\}_{f\in E^*})$. As $f\in E^*$ is
continuous with respect to the regular topology, the topology $\sigma(E, E^*)$
is coarser then the regular topology, which we call strong.
\vspace{1ex}

On the space $E^*$, we also have strong topology with the operator norm.
Additionally, we have the weak-$*$ topology $\sigma(E^*, E)=\tau(\{v\}_{v\in E})$.

\vspace{1ex}
\textbf{Proposition:} \textit{The weak-$*$ topology is a trace topology from the space
$\mathbb{R}^E$ with the product topology.}

\vspace{1ex}
\textbf{Proof:} Let $\tau(\{\pi_v\}_{v\in E})$ be the trace topology. Then it
is easy to see that $\pi_v=v$ as both function are evaluations at $v$ and thus
$\tau(\{\pi_v\}_{v\in E})=\tau(\{v\}_{v\in E})=\sigma(E^*, E)$ is a weak-$*$
topology.

\vspace{1ex}
\textbf{Remark:} In the book "Functional Analysis" by Haim Brezis, the part
above is done by establishing an homeomorpism and a verification of its bicontinuity.
As you have seen, there is actually nothing substantial to prove since these are 
just two notions of the same concept – projection and evaluation in the dual-space.

\vspace{1ex}
\textbf{Theorem (Banach-Alaoglu):} \textit{The closed unit ball $B=\{f\in E^*\,|\,|
f|\leq 1\}$ is compact in the weak-$*$ topology $\sigma(E^*, E)$.}

\vspace{1ex}
\textbf{Proof:}
\[ B=\left\{f\in\mathbb{R}^E\,|\,
\begin{cases}
    |f(x)|<|x|,\;\forall x\in E\\
    f(\lambda x)=\lambda f(x),\;\forall\lambda\in\mathbb{R}, x\in E\\
    f(x+y)=f(x)+f(y)\;\forall x,y\in E
\end{cases}
\right\} \] 

Hence it is intersection of the following sets $B=K\cap\bigcap_{x,y\in E} A_{x,y}
\cap\bigcap_{x\in E, \lambda\in\mathbb{R}}B_{\lambda,x}$, where $K=\{f\in\mathbb
{R}^E\,|\,|f(x)|\leq|x|\}=\prod_{x\in E}[-|x|, |x|]$ is compact by Tichonoff
theorem, where for $x,y\in E$, we define $A_{x,y}=\{f\in\mathbb{R}^E\,|\,f(x+y)-
f(x)-f(y)=0\}$, which is closed since evaluations and addition are continuous, and
thus $f\mapsto f(x+y)-f(x)-f(y)$ is continuous and $A_{x,y}$. For similar
reasons $B_{\lambda, x}=\{f\in\mathbb{R}^E\,|\,f(\lambda x)-\lambda f(x)=0\}$ is
closed. This proves that $B$ is compact.

\subsection{Vector valued measure}
Later on, when we'll need to talk about generalisation of weak derivatives and
for this purpose we'll need to have the theory in a more generic context. More
precicely we'll need vector valued measures and then if needed, we can take a
restriction to Radon measures.

\vspace{2ex}
Let $X$ be a topological space and $V$ a Banach space, then $\mu:\mathcal{B}(X)
\rightarrow V$ is a $V$-valued Borel measure if
\[\sum_n\mu(E_n)=\mu(\bigcup_n E_n)\]
for any disjoint countable family $\{E_n\}$ of Borel sets. From that definition
we have $\mu(A)+\mu(\varnothing)=\mu(A\cup\varnothing)=\mu(A)$ and thus 
$\mu(\varnothing)=0$. This is a quite a strong property as the convergence of
the sum does not depend on the order, which in finite dimenetions is equivalent
to the absolute convergence of that series.

\vspace{1ex} Let $\mu$ be a vector valued measure. Then the \emph{total
variation} $|\mu|$ of a Borel set $A$  by measure $\mu$ is defined by:
\[|\mu|(A) = \text{sup}\{\sum_n|\mu(A_n)|\,|\,\{A_n\}\text{ countable partition of }A\}\]

\textbf{Proposition:} \textit{Total variation is a positive bounded measure.}

\vspace{1ex}
It is easy to see that $|\mu|(\varnothing)=0$ since all partitions of an empty
set consist of empty sets which measure is zero. The image of $|\mu|$ by the
definition consists of positive numbers. Lastly we verify $\sigma$-additivity.
Let $\{S_n\}$ be a disjoint countable collection of Borel sets. Then
\[ 
    \sum_n|\mu|(S_n) = \sum_n\sup\{\sum_m|\mu(S_{n,m})|\,|\,(S_{n,m})_m\text{ is a countable Borel partition of }S_n\} \\ 
\]
Then we remark that for each choice of $\{S_{n,m}\}$, it is a countable Borel
partition of $S=\bigcup_n S_n$, and thus $|\mu|(S)\geq\sum_n|\mu|(S_n)$. On the
other hand if $\{A_k\}$ is a countable Borel partition of $S$ then we have
partitions of $S_n$ defined as $\{S_{n,k}=A_k\cap S_n\}_k$ and we have the
following inequality:
\[
    \sum_k|\mu(A_k)|=\sum_k|\sum_n\mu(S_{n,k})|\leq\sum_n\sum_k|\mu(S_{n,k})|
\]
which implies $|\mu|(S)\leq\sum_n|\mu|(S_n)$ and we conclude that $|\mu|$ is a
positive measure.

\vspace{1ex}
Let's verify that total variation is bounded.
That is a tricker question and we
shall follow the proof from "...". The measure can be partitioned into projection measures
$\mu=(\mu_i)_{i=1}^n$. As all the norms are equivalent we can concider $|\cdot|
= \|\cdot\|_1$. Then as we have the following inequality:
\[\sup\{\sum_i|\mu(X_i)|\,|\,X_i\text{ is a borel partition of }X\} \leq \sum_j\sup\{\sum_i|\mu_j(X_i)|\,|\,X_i\text{ is a borel partition of }X\}\]
It is sufficient to prove that for real valued measures its total variation is bound.
If we suppose it is not, then we have a real valued measure $\mu$,
countable Borel partition of $X$ $\{X_m\}_m$ and $n\in\mathbb{N}$ such that
\[\sum_{m=0}^n|\mu(X_m)|>2(|\mu(X)|+1)\]
Let $P=\{X_i|\mu(X_i)>0\}$ and $N=\{X_i|\mu(X_i)<0\}$. Then
we have $|\mu(\bigcup P)|>|\mu(X)|+1$ or $|\mu(\bigcup N)|>|\mu(X)|+1$, thus we
have a set $E$ such that $|\mu(E)|>|\mu(X)|+1$. Then we have $|\mu(E^c)|=|\mu(X)
-\mu(E)|\geq |\mu(E)|-|\mu(X)|>1$. Then by additivity of $|\mu|$ we have
$|\mu|(E)=\infty$ or $|\mu|(E^c)=\infty$; supposing the latter we pose $E_1=E$
(or $=F$) we always have $\mu(E_1)>1$ and if we continue the same procidure for
$X=E^c$ we construct by the choice axiom the following sequence of disjoint sets
$(E_i)_i$ and $|\mu|(E_i)>1$ and thus $\sum\mu(E_i)$ does not converge and we
have a contracdiction to the definition of vector valued measure. Thus $\mu$
is bound.

\vspace{1ex}
By setting
\[\mu_+=\frac{|\mu|+\mu}{2}\quad\quad\quad\quad\quad\quad\mu_-=\frac{|\mu|-\mu}{2}\]
we have $\mu_+$ and $\mu_-$ positive bounded measures and $\mu=\mu_+-\mu_-$
which ports a name a \emph{Jordan decomposition}.


\vspace{1ex}
The \emph{mass} of $\mu$ is set to be $\|\mu\|=|\mu|(X)$.

\vspace{1ex}
\textbf{Proposition:}
\textit{The set of vector norms with the mass form a normed vector space.}

\vspace{1ex}
\textbf{Proof:} Let $\mu:\mathcal{B}(X)\rightarrow V$ for $V$ an $\mathbb
R$-vector space be a vector norm. Then evidently $\|k\mu\|=|k|\|\mu\|$. Let $\nu$
be another vector measure then
\begin{align*}
    \|\mu+\nu\|&=\sup\left\{\sum_{n=0}^{+\infty}|(\mu+\nu)(E_n)|\,|\,\{E_n\}_n\text{ countable paritition of }X\right\}\\
    &\leq\sup\left\{\sum_{n=0}^{+\infty}|\mu(E_n)|+|\nu(E_n)|\,|\,\{E_n\}_n\text{ countable paritition of }X\right\}\\
    &\leq\sup\left\{\sum_{n=0}^{+\infty}|\mu(E_n)|\,|\,\{E_n\}_n\text{ countable paritition of }X\right\}\\
    &+\sup\left\{\sum_{n=0}^{+\infty}|\nu(E_n)|\,|\,\{E_n\}_n\text{ countable paritition of }X\right\}\\
    &=\|\mu\|+\|\nu\|
\end{align*}

\subsection{Riesz representation theorems for vector valued measure}
For an $\mathbb{R}^n$-valued measure $\mu$ on $X$ we define an associated
functional
\begin{align*}
\Lambda_\mu:\mathcal C_0(X,\mathbb{R}^n)&\rightarrow\mathbb{R}\\
f&\mapsto\int f\,d\mu
\end{align*}

\textbf{Riesz representation theorem:} \textit{
The map
\begin{align*}
\Lambda:\mathcal{M}(X, \mathbb{R}^n)&\rightarrow\mathcal{C}_0(X,\mathbb{R}^n)^*\\
\mu\quad&\mapsto\quad\Lambda_\mu
\end{align*}
is an isometry}

\vspace{1ex}
\textbf{Proof:} The injectivity of $\Lambda$ is quit obvious. For sujectivity
we make an inverse construction, for a given functional $L$ we take its total
variation defined by
\[|L|(A)=\sup\{\langle L\,|\,\phi\rangle\,|\,\phi\in\mathcal C_c(A,\mathbb R^n), |\phi|<1\}\]
for open set $A$. And for other sets we set
\[|L|(E)=\inf\{|L|(A)\,|\,E\subseteq A\}\]
Thus $|L|$ is localy finite, because it's continuous and thus bounded. Whats
more the second property yeilds us the regularity of the total variation as we
can take a countable intersection of $\{A_n\}$ of open sets such that $|L|(A_n)
\rightarrow |L|(E)$, thus the total vatiation is a radon measure. Then the
proof of existance of function $f$ such that $L$ is equal to integration with
respect to $f|L|$ and $|f|=1$ $|L$-a.e. can be found on pages 34-41
\cite{maggi}. Let's check that it's an isometry. Let measure $\mu$ be
represented by a functional $L$. Then 
\begin{align*}
    &\|L\|=\sup\{L(f)\;|\;\|f\|\le1\}\\
    &\|\mu\|=\sup\{\sum|\mu(B_i)|\,|\,\{B_i\}\text{ a partition of } X\}
\end{align*}

For $f\in C_0(X,\mathbb R^n)$ we find a series of step functions $f_n=\sum a_i\chi_{B_i}$ 
such that $f_n\rightarrow f$ and $a_i\leq 1$. Thus by dominant convergence
$\int f_n\,d\mu\rightarrow L(f)$. On the other hand we have
\[|\int f_n\,d\mu = |\sum a_i\cdot\mu(B_i)|\leq\sum|a_i||\mu(B_i)|\leq\sum|\mu(B_i)|\leq\|\mu\|\]

and thus we have $|\int f\,d_\mu|\leq\|\mu\|$ and thus $\|L\|\leq\|\mu\|$.

In the other direction it obviously follows form the fact the $C^0$ is dense
in $L^1$.

\vspace{2ex}
\textbf{Corollary:} \textit{Continuous positive funcitonals are represented by
Radon measures}

\vspace{1ex}
Because $f=1$.

\subsection{Interpretation of Banach-Alaoglu theorem for vector valued measures}
The weak-$*$ convergence can be interpreted as convergence of evaluation of measure
on every continuous function on compact sets.

\vspace{2ex}
The original statement of Banch-Alaoglu theorem is \textbf{the closed unit ball
$B=\{f\in E^*\,|\,|f|\leq 1\}$ is compact in the weak-$*$ topology}. If we replace
the termes in this proof by measure terms we have the following theorem

\vspace{1ex}
\textbf{Banach-Alaoglu Theorem for $\mathcal M(X,\mathbb{R}^n)$:}\textit{
The set $B=\{\mu\in\mathcal M(X,\mathbb{R}^n)\,|\,\|\mu\|\leq C\}$ is compact
for every $C\in\mathbb{R}_{>0}$. That's said every bounded sequence of vector
measures has a weakly-$*$ converging subsequence.}

\vspace{1ex}
\textbf{Consequence:} If $(\mu_n)$ is a bounded sequence of vector measures, 
then it has a converging subsequence.

\subsection{Weak-* convergence of measures}
\textbf{Proposition:} \textit{Let $(\mu_n)_n$ be a sequence of positives
measures converging to $\mu$, then we have
\begin{enumerate}
    \item For any open subset $A\subseteq X$, $\liminf\mu_n(A)\geq\mu(A)$
    \item For any compact subset $K\subseteq X$, $\limsup\mu_n(K)\leq\mu(K)$
    \item For any relatively compact $E\subseteq X$ such that $µ(\partial E)=0$
\end{enumerate}
}

\vspace{1ex}
\textbf{Proof:} We will simultaneously demonstrate propositions 1 and 2. Let $K \subset A$,
where $K$ is compact and $A$ is open. Consider a function $f\in\mathcal{C}
_c(X)$ such that $\chi_K\leq f\leq\chi_A$. For a Radon measure $\nu$, we then have:
\[\nu(K) \leq \int f \,d\nu \leq \nu(A)\]
And by considering the limits, we obtain:
\[\limsup \mu_i(K) \leq \limsup \int f \,d\mu_i = \int f \,d\mu \leq \mu(A)\]
\[\mu(K) \leq \int f \,d\mu = \liminf \int f \,d\mu_i \leq \liminf \mu_i(A)\]
Since we are dealing with Radon measures and these inequalities hold for
every compact $K$ and every open $A$, we can pass to the limit. The
lines then transform into:
\[\limsup \mu_i(K) \leq \mu(K)\]
\[\mu(A) \leq \liminf \mu_i(A)\]
Point three is a consequence of the two preceding points. Indeed, we have:
\[\limsup\mu_i(\overline E)\leq\mu(\overline E)=\mu(\text{int}(E))\leq\liminf\mu_i(\text{int}(E))\]

\vspace{2ex}
\textbf{Remark:} The third proposition gives us the property that the
measures of a sequence of figures converge to the other figure; thus,
locally in the ball, the area also converges.
